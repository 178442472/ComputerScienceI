\documentclass[12pt]{scrartcl}

\usepackage[printwatermark,disablegeometry]{xwatermark}

\usepackage{epsfig,amssymb}

\usepackage{xcolor}
\usepackage{graphicx}
\usepackage{epstopdf}
\usepackage{multirow}

\definecolor{darkred}{rgb}{0.5,0,0}
\definecolor{darkgreen}{rgb}{0,0.5,0}
\usepackage{hyperref}
\hypersetup{
  letterpaper,
  colorlinks,
  linkcolor=red,
  citecolor=darkgreen,
  menucolor=darkred,
  urlcolor=blue,
  bookmarks=true,
  pdfpagemode=none,
  pdftitle={},
  pdflang={en},
  pdfkeywords={}
}

\usepackage{fullpage}
\usepackage{tikz}
\pagestyle{empty} %
\usepackage{subfigure}

\definecolor{MyDarkBlue}{rgb}{0,0.08,0.45}
\definecolor{MyDarkRed}{rgb}{0.45,0.08,0}
\definecolor{MyDarkGreen}{rgb}{0.08,0.45,0.08}

\definecolor{mintedBackground}{rgb}{0.95,0.95,0.95}
\definecolor{mintedInlineBackground}{rgb}{.90,.90,1}

\usepackage[newfloat=true]{minted}

\setminted{mathescape,
           linenos,
           autogobble,
           frame=none,
           framesep=2mm,
           framerule=0.4pt,
           %label=foo,
           xleftmargin=2em,
           xrightmargin=0em,
           %startinline=true,  %PHP only, allow it to omit the PHP Tags *** with this option, variables using dollar sign in comments are treated as latex math
           numbersep=10pt, %gap between line numbers and start of line
           style=default} %syntax highlighting style, default is "default"

\setmintedinline{bgcolor={mintedBackground}}
%doesn't work with the above workaround:
\setminted{bgcolor={mintedBackground}}
\setminted[text]{bgcolor={mintedBackground},linenos=false,autogobble,xleftmargin=1em}
%\setminted[php]{bgcolor=mintedBackgroundPHP} %startinline=True}
\SetupFloatingEnvironment{listing}{name=Code Sample}
\SetupFloatingEnvironment{listing}{listname=List of Code Samples}

\setlength{\parindent}{0pt} %
\setlength{\parskip}{.25cm}
\newcommand{\comment}[1]{}

\usepackage{amsmath}
\usepackage{algorithm2e}
\SetKwInOut{Input}{input}
\SetKwInOut{Output}{output}
%NOTE: you can embed algorithms in solutions, but they cannot be floating objects; use [H] to make them non-floats

\usepackage{lastpage}

%\usepackage{titling}
\usepackage{fancyhdr}
\renewcommand*{\titlepagestyle}{fancy}
\pagestyle{fancy}
%\renewcommand*{\titlepagestyle}{fancy}
%\fancyhf{}
%\rhead{Computer Science I}
%\lhead{Guides and tutorials}
\renewcommand{\headrulewidth}{0.0pt}
\renewcommand{\footrulewidth}{0.4pt}
\lfoot{CL/LA Guidelines -- Computer Science I}
\cfoot{~}
\rfoot{\thepage\ / \pageref*{LastPage}}

\makeatletter
\title{Guidelines for Course Leaders \& Learning Assistants}\let\Title\@title
\subtitle{Computer Science I\\
{\small
\vskip1cm
Department of Computer Science \& Engineering \\
University of Nebraska--Lincoln}
\vskip-3cm}
%\author{Dr.\ Chris Bourke}
\date{~}
\makeatother

\begin{document}

\maketitle

%\newwatermark[allpages=true,scale=5,textmark=Draft]{},

\hrule

\section*{Overview}

The instructor sets policies in the syllabus which all students 
are expected to read, understand and adhere to. Every Course Leader
(CL) and Learning Assistant (LA) is expected 
to read, understand and also follow these policies.  Often, 
students will attempt to violate these policies or ask for special 
consideration.  Do not speculate or otherwise discuss possible 
exceptions to these policies.  Direct them to the instructor 
and follow up with the instructor if necessary.

\begin{itemize}
  \item Be prepared.  Be aware of the course content and expectations.
  You are responsible for knowing the material so that you can effectively
  explain and demonstrate it to students.  Be able to complete the assignments, 
  labs, etc.\ yourself.  If you have doubts or concerns engage first with
  your Course Leader (CL) supervisor and/or other LAs.  If you cannot
  resolve the issue, ask the instructor for clarification.
  \item Manage your time.  You have made a commitment to this course and
  will be expected to fulfill it.  Work and plan ahead.  Be aware of 
  upcoming due dates in this course as well as your own courses, research, 
  personal obligations, etc.  Plan ahead and make appropriate accommodations 
  if you know there will be an excess of work during a period of time.
\end{itemize}

\subsection*{Course Structure}

This course is structured with a single large lecture section with the
capacity to enroll nearly 300 students.  Despite the size, it is our 
goal to foster a greater sense of community among these students in 
our department and in our discipline.  

The course is a traditional CS1 course covering basic CS1 topics 
using the C programming language (offered as our traditional CSCE 155E).  
However, in addition to a traditional lecture, we've produced dozens of 
lecture/tutorial videos for students to view before and/or after lecture.  
We also have extensive required reading (mostly from my free textbook but 
also supplemental resources).  

We have doubled the number of weekly lab sessions meeting on Tuesdays and
Thursdays.  On Tuesdays, students are paired up and expected to complete 
several peer programming exercises.  They are 
expected to complete the labs in the lab time and are graded only on 
completion.

The second section (Thursdays) is a ``hack session'' in which they are 
allowed (and encouraged) to collaborate with as many other students as 
they wish.  They complete small programs or pieces of code that are 
then submitted to our online grading system.  The hacks are due the 
following Monday at midnight.  In addition, there are
be several (usually 5) programming assignments each student is expected
to complete individually.  There is also 1 midterm and 1 final (both 
are open book/note/computer and require live programming exercises). 

\subsubsection*{Online Sections}

This course has multiple sections:
\begin{itemize}
  \item Section 250 is the traditional in-person section with in-person
  labs and hacks, on-campus testing (during regular lecture time), etc.
  \item Section 700 is the online section available for all matriculating
  students (with no on-campus testing)
  \item Section 800 is the online section for Nebraska Now (no on-campus
  testing)
\end{itemize}

\subsection*{General Responsibilities}

Learning Assistants will have several responsibilities in addition to the
responsibilities and expectations of the Learning Assistant program.

\begin{itemize}
  \item Holding several office hours per week in a designated area
  \item Assisting in 1--3 weekly lab and hack sessions
  \item Grading hacks, assignments and exams
  \item Mentoring and helping students in additional office hours, hack sessions
  and online via Piazza
  \item General administrative duties (entering grades, paperwork, etc.) as needed
  \item Other duties may include course development, materials development 
  (solution keys, future exercises, etc.) and other tasks identified by the instructor.
\end{itemize}

Course Leaders will have the following general responsibilities
\begin{itemize}
  \item Supervising lab/hack sessions and assisting students in them
  \item Supervise grading and ensure that all assignments are graded 
    in a timely manner
  \item Providing quality control (QC) by checking grading and ensuring consistency
  \item Holding regular office hours in a designated area
  \item Be in regular communication and hold weekly meetings with the instructor
\end{itemize}


\section*{Communication}

\begin{itemize}
  \item Piazza is our primary means of communication; use it and encourage 
students to use it.  
  \item If you receive email from students, answer it, 
but redirect them in the future to Piazza.  If the question/answer
would be of benefit to the class as a whole, post the question/answer
to Piazza and inform the student they can find the answer there.
  \item For communications among instructor(s), CLs and LAs, use Piazza but
make it a private message, viewable only to TAs/instructors.
  \item If a question has been asked/answered before, link to the original
  post as your answer.  
  \item Be professional in all your communications, be courteous and
  helpful.  
  \item Be prompt in answering communications.  No question or email should go 
  unanswered for more than 24 business hours.\footnote{Within 24 hours but only
  on business days, i.e.\ excluding weekends and holidays}
\end{itemize}

\subsection*{Weekly Coordinating Meetings}

Each week, we will hold regularly scheduled coordination meetings.  CLs are
required to attend and LAs are welcome to attend.
These meetings are intended to gather and disseminate information about the past
week and upcoming items.  In particular, CLs should be prepared to report and
discuss the following:
\begin{itemize}
  \item How well lab/hack went: what went well, what didn't go so well and any
  suggestions on improvements.
  \item Lab attendance will be gauged via the grade book, but Hack attendance 
  should be recorded and reported.
  \item You should keep an estimate of how much time it took for 50\% of the
  students to complete the lab.
  \item You should keep an estimate on office hour attendance and the common
  issues/questions that students are having.
  \item Share any other information on the course as you see fit
\end{itemize}

\section*{Grading}

\subsubsection*{Timeline}

\begin{itemize}
  \item Hacks are generally due on Monday evenings at midnight.  
    Randomized grading assignments are to 
    be posted to Piazza prior to the due date/time by the instructor.
  \item Learning Assistants are required to have completed their assigned
    grading by 5PM 2 business days after each assignment due date.
    Upon completion, Learning Assistants should continue to available via
    Piazza/email for any issues that need to be resolved.
  \item Course Leaders should have everything reviewed and
    any issues resolved by 5PM 48 business hours later at which time grades
    will be released to students.
  \item If Learning Assistants face any impediments or issues to completing
    their grading on time, they should discuss this immediately with their CL
    supervisor or instructor.  A Course Leader can then be called
    in ``off the bench'' to assist with grading that week.
\end{itemize}

\subsubsection*{Instructions}

\begin{itemize}
  \item Grading is done through a combination of the online webgrader system
    and codepost.    
  \item Time is limited and it should not be wasted trying to troubleshoot 
    code that won't compile or run.  If the code is ungradeable or does not
    compile/run then take at most 5 minutes to look over
    the code.  If the issue can be fixed within that time frame, back up the
    original, fix it, note the differences (via code comments) and grade 
    accordingly.  If you cannot resolve the issue within 5 minutes, assign
    the student a zero and move on.  This will require you to login to the
    command line and edit the files directly.  Note that the original copy
    stored in the webhandin system will remain.
  \item Grade in accordance to the rubric through codepost.  If the rubric 
    does not address something or there is a \emph{reasonable} uncertainty, 
    discuss it with your CL supervisor.
  \item Grade in a consistent manner, both between individual assignments and
    with other graders.  There should not be a significant variation in
    points deducted or awarded for similar mistakes or work.  Consistency 
    and grading quality will be checked by your CL supervisor.
  \item When you deduct points, give clear and reasonably detailed reasons
    and justifications for doing so.  Good feedback is essential for the
    students' learning experience.  Put in efforts to provide constructive
    feedback and positive feedback for good work.
  \item Make a comment on \emph{every} assignment you grade so that the students
    will know who graded it.  This comment should be substantial and positive 
    regardless of how the student performed.  If poorly, suggest strategies 
    for improvement.  If they did well, congratulate them.  Document any further
    interaction or changes with a comment as well.
  \item In general, unless otherwise stated, the formatting of output is
    left up to the student.  As long as output formatting is reasonable and
    conveys \emph{just as much} information as the expected output, it should
    be graded as correct.
\end{itemize}

\section*{Vigilance}

\begin{itemize}
  \item Be on the lookout for improvements to policies, processes, 
    grading, course material, etc.  I welcome any and all feedback and would
    appreciate it.
  \item Be on the lookout for suspected academic integrity violations, odd
    code idiosyncrasies or patterns not covered in class, similarities in code, 
    disparate performance in class/lab/hack and grades received, etc.  However,
    never confront a student directly.  Bring your concerns to the instructor and
    be sure to document everything.
  \item Be on the lookout for racial, gender, or other biases or incidents.  
    It is essential that we promote an open and equitable environment for everyone.
    If you see a potential issue or event, please intervene and correct it immediately.
    Report incidences to the instructor as soon as possible.  
\end{itemize}

\section*{Online Office Hours}

\begin{itemize}
  \item Your instructor will provide you with a ``standing meeting'' 
  zoom (\url{https://unl.zoom.us/}) link or meeting ID to use 
  \item You will need a reliable internet connection, microphone and 
  (optionally) a webcam.
  \item You should wear headphones to reduce feedback and hold your 
  office hours in a relatively quiet/noise free area
  \item When you join a meeting:
  \begin{itemize} 
    \item In the ``more'' drop down, set it so that zoom plays a chime on 
    enter/exit so that if you are running zoom in the background you are 
    made aware when a student joins
    \item Maximize the chat and ``manage participants'' windows so you can 
    see everyone and are aware of the chat contents
  \end{itemize}
  \item In general, mute and stop your video unless you need to speak
  \item It is suggested you get a physical webcam cover and use it to protect your privacy
  \item Upload an appropriate photo of yourself to your profile so students can identify you when you are not streaming video
  \item Rename your profile name to what you use with students and include an \mintinline{text}{(LA)} tag in it so students can identify that you are an LA 

  \item When a student enters, be sure to address them and ask how you can help right away.  Notify them politely if they are muted
  \item Your instructor will provide you with a \emph{host key} that allows you to claim host duties if there is no host already.
  \item As host, you will be tasked with the following additional responsibilities:
  \begin{itemize}
    \item Create breakout rooms and assign students/LAs to them
    \item Note: You \emph{may} need to establish a number of breakout rooms (create 20 and select manual assignments) if no rooms have been created as, once breakout rooms have been established, no additional rooms can be created until all have been removed
    \item Try to stay in the main room or hand off hosting duties to another LA before you go to a breakout room otherwise keep the breakout room window open so you know when others enter the main room as new students joining a lobby will only see those currently in the lobby and may assume no LAs are available
    \item Hand off host duties to another LA (if applicable) before you leave. Leaving without first reassigning host may result in a student being assigned host responsibilities
    \item Be sure to rotate LAs that you ask to help with students
  \end{itemize}
\end{itemize}


\end{document}



