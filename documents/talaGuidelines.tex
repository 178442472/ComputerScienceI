\documentclass[12pt]{scrartcl}

\usepackage[printwatermark,disablegeometry]{xwatermark}

\usepackage{epsfig,amssymb}

\usepackage{xcolor}
\usepackage{graphicx}
\usepackage{epstopdf}
\usepackage{multirow}

\definecolor{darkred}{rgb}{0.5,0,0}
\definecolor{darkgreen}{rgb}{0,0.5,0}
\usepackage{hyperref}
\hypersetup{
  letterpaper,
  colorlinks,
  linkcolor=red,
  citecolor=darkgreen,
  menucolor=darkred,
  urlcolor=blue,
  bookmarks=true,
  pdfpagemode=none,
  pdftitle={BYOD: Bring Your Own Device},
  pdflang={en},
  pdfauthor={Christopher M. Bourke},
  pdfcreator={$ $Id: cv-us.tex,v 1.28 2009/01/01 00:00:00 cbourke Exp $ $},
  pdfsubject={PhD Thesis},
  pdfkeywords={}
}

\usepackage{fullpage}
\usepackage{tikz}
\pagestyle{empty} %
\usepackage{subfigure}

\definecolor{MyDarkBlue}{rgb}{0,0.08,0.45}
\definecolor{MyDarkRed}{rgb}{0.45,0.08,0}
\definecolor{MyDarkGreen}{rgb}{0.08,0.45,0.08}

\definecolor{mintedBackground}{rgb}{0.95,0.95,0.95}
\definecolor{mintedInlineBackground}{rgb}{.90,.90,1}

\usepackage[newfloat=true]{minted}

\setminted{mathescape,
           linenos,
           autogobble,
           frame=none,
           framesep=2mm,
           framerule=0.4pt,
           %label=foo,
           xleftmargin=2em,
           xrightmargin=0em,
           %startinline=true,  %PHP only, allow it to omit the PHP Tags *** with this option, variables using dollar sign in comments are treated as latex math
           numbersep=10pt, %gap between line numbers and start of line
           style=default} %syntax highlighting style, default is "default"

\setmintedinline{bgcolor={mintedBackground}}
%doesn't work with the above workaround:
\setminted{bgcolor={mintedBackground}}
\setminted[text]{bgcolor={mintedBackground},linenos=false,autogobble,xleftmargin=1em}
%\setminted[php]{bgcolor=mintedBackgroundPHP} %startinline=True}
\SetupFloatingEnvironment{listing}{name=Code Sample}
\SetupFloatingEnvironment{listing}{listname=List of Code Samples}

\setlength{\parindent}{0pt} %
\setlength{\parskip}{.25cm}
\newcommand{\comment}[1]{}

\usepackage{amsmath}
\usepackage{algorithm2e}
\SetKwInOut{Input}{input}
\SetKwInOut{Output}{output}
%NOTE: you can embed algorithms in solutions, but they cannot be floating objects; use [H] to make them non-floats

\usepackage{lastpage}

%\usepackage{titling}
\usepackage{fancyhdr}
\renewcommand*{\titlepagestyle}{fancy}
\pagestyle{fancy}
%\renewcommand*{\titlepagestyle}{fancy}
%\fancyhf{}
%\rhead{Computer Science I}
%\lhead{Guides and tutorials}
\renewcommand{\headrulewidth}{0.0pt}
\renewcommand{\footrulewidth}{0.4pt}
\lfoot{GTA/LA Guidelines -- Computer Science I}
\cfoot{~}
\rfoot{\thepage\ / \pageref*{LastPage}}

\makeatletter
\title{Guidelines for Graduate Teaching Assistants \& Learning Assistants}\let\Title\@title
\subtitle{Computer Science I\\
{\small
\vskip1cm
Department of Computer Science \& Engineering \\
University of Nebraska--Lincoln}
\vskip-1cm}
%\author{Dr.\ Chris Bourke}
\date{~}
\makeatother

\begin{document}

\maketitle

\newwatermark[allpages=true,scale=5,textmark=Draft]{},

\hrule

\section*{Overview}

The instructor sets policies in the syllabus which all students 
are expected to read, understand and adhere to. Every Graduate 
Teaching Assistant (GTA) and Learning Assistant (LA) is expected 
to read, understand and also follow these policies.  Often, 
students will attempt to violate these policies or ask for special 
consideration.  Do not speculate or otherwise discuss possible 
exceptions to these policies.  Direct them to the instructor 
and follow up with the instructor if necessary.

\begin{itemize}
  \item Be prepared.  Be aware of the course content and expectations.
  You are responsible for knowing the material so that you can effectively
  explain and demonstrate it to students.  Be able to complete the assignments, 
  labs, etc.\ yourself.  If you have doubts or concerns engage first with
  other GTA/LAs and then ask the instructor for clarification.
  \item Manage your time.  You have made a commitment to this course and
  will be expected to fulfill it.  Work and plan ahead.  Be aware of 
  upcoming due dates in this course as well as your own courses, research, 
  personal obligations, etc.  Plan ahead and make appropriate accommodations 
  if you know there will be an excess of work during a period of time.
\end{itemize}

\section*{Communication}

\begin{itemize}
  \item Piazza is our primary means of communication, use it and encourage 
students to use it.  
  \item If you receive email from students, answer it, 
but redirect them in the future to Piazza.  If the question/answer
would be of benefit to the class as a whole, post the question/answer
to Piazza and inform the student they can find the answer there.
  \item For communications among instructor(s), GTAs and LAs, use Piazza but
make it a private message, viewable only to TAs/instructors.
  \item If a question has been asked/answered before, link to the original
  post as your answer.
  \item Be professional in all your communications, be courteous and
  helpful.  
  \item Be prompt in answering communications.  No question or email should go 
  unanswered for more than 24 business hours.\footnote{Within 24 hours but only
  on business days, i.e.\ excluding weekends and holidays}
\end{itemize}

\section*{Grading}

\subsection*{Timeline}

\begin{itemize}
  \item Assignments and weekly Hacks are due on Fridays at midnight.  
    Grading assignments should be sent out prior to the due date/time.
  \item Learning Assistants are required to have completed their assigned
    grading by 5PM the following Tuesday (or within 48 business hours of
    the due date).  Upon completion Learning
    Assistants should notify their GTA supervisor and be available via
    email for any issues that need to be resolved.
  \item Graduate Teaching Assistants should have everything reviewed and
    any issues resolved by 5PM the following Thursday at which time 
    grades will be released to students.
  \item If Learning Assistants face any impediments or issues to completing
    their grading on time, they should discuss this with their GTA supervisor
    who will be responsible for helping to resolve the issue by either 
    temporarily helping with grading or shifting grading assignments.  
    If a GTA cannot resolve the issue, they should consult with the 
    instructor.
\end{itemize}

\subsection*{Directives}

\begin{itemize}
  \item All grading is done through the online webgrader system.  
  \item Time is limited and it should not be wasted trying to troubleshoot 
    code that won't compile or run.  Take, at most, 5 minutes to look over
    the code.  If the issue can be fixed within that time frame, back up the
    original, fix it, note the differences (via code comments) and grade 
    accordingly.  If you cannot resolve the issue within 5 minutes, assign
    the student a zero and move on.  This will require you to login to the
    command line and edit the files directly.  Note that the original copy
    stored in the webhandin system will remain.
  \item Grade in accordance to the rubric through Canvas.  If the rubric 
    does not address something or there is a \emph{reasonable} uncertainty, 
    discuss it with your GTA supervisor.
  \item Grade in a consistent manner, both between individual assignments and
    with other graders.  There should not be a significant variation in
    points deducted or awarded for similar mistakes or work.  Consistency 
    and grading quality will be checked by your GTA supervisor.
  \item When you deduct points, give clear and reasonably detailed reasons
    and justifications for doing so.  Good feedback is essential for the
    students' learning experience.  Put in efforts to provide constructive
    feedback and positive feedback for good work.
  \item The online rubric in Canvas should indicate your name to the student, 
    but just in case, clearly indicated it in the comments.  Add comments to
    make any notes on changes or other administrative items (corrections, 
    regrades, etc.)
\end{itemize}



\end{document}
