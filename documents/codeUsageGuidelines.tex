\documentclass[12pt]{scrartcl}


\usepackage{epsfig,amssymb}

\usepackage{xcolor}
\usepackage{graphicx}
\usepackage{epstopdf}
\usepackage{multirow}

\definecolor{darkred}{rgb}{0.5,0,0}
\definecolor{darkgreen}{rgb}{0,0.5,0}
\usepackage{hyperref}
\hypersetup{
  letterpaper,
  colorlinks,
  linkcolor=red,
  citecolor=darkgreen,
  menucolor=darkred,
  urlcolor=blue,
  pdfpagemode=none,
  pdftitle={Code Usage Guidelines},
  pdfkeywords={}
}

\usepackage{fullpage}
\usepackage{tikz}
\pagestyle{empty} %
\usepackage{subfigure}

\definecolor{MyDarkBlue}{rgb}{0,0.08,0.45}
\definecolor{MyDarkRed}{rgb}{0.45,0.08,0}
\definecolor{MyDarkGreen}{rgb}{0.08,0.45,0.08}

\definecolor{mintedBackground}{rgb}{0.95,0.95,0.95}
\definecolor{mintedInlineBackground}{rgb}{.90,.90,1}

\usepackage[newfloat=true]{minted}

\setminted{mathescape,
           linenos,
           autogobble,
           frame=none,
           framesep=2mm,
           framerule=0.4pt,
           %label=foo,
           xleftmargin=2em,
           xrightmargin=0em,
           %startinline=true,  %PHP only, allow it to omit the PHP Tags *** with this option, variables using dollar sign in comments are treated as latex math
           numbersep=10pt, %gap between line numbers and start of line
           style=default} %syntax highlighting style, default is "default"

\setmintedinline{bgcolor={mintedBackground}}
%doesn't work with the above workaround:
\setminted{bgcolor={mintedBackground}}
\setminted[text]{bgcolor={mintedBackground},linenos=false,autogobble,xleftmargin=1em}
%\setminted[php]{bgcolor=mintedBackgroundPHP} %startinline=True}
\SetupFloatingEnvironment{listing}{name=Code Sample}
\SetupFloatingEnvironment{listing}{listname=List of Code Samples}

\setlength{\parindent}{0pt} %
\setlength{\parskip}{.25cm}
\newcommand{\comment}[1]{}

\usepackage{amsmath}
\usepackage{algorithm2e}
\SetKwInOut{Input}{input}
\SetKwInOut{Output}{output}
%NOTE: you can embed algorithms in solutions, but they cannot be floating objects; use [H] to make them non-floats

\usepackage{lastpage}

%\usepackage{titling}
\usepackage{fancyhdr}
\renewcommand*{\titlepagestyle}{fancy}
\pagestyle{fancy}
%\renewcommand*{\titlepagestyle}{fancy}
%\fancyhf{}
%\rhead{Computer Science I}
%\lhead{Guides and tutorials}
\renewcommand{\headrulewidth}{0.0pt}
\renewcommand{\footrulewidth}{0.4pt}

\lhead{~}
\chead{~}
\rhead{~}
\lfoot{\Title}
\cfoot{~}
\rfoot{\thepage\ / \pageref*{LastPage}}

\makeatletter
\title{Academic Integrity \& Code Usage Guidelines}\let\Title\@title
\subtitle{Computer Science I\\
{\small
\vskip.5cm
Department of Computer Science \& Engineering \\
University of Nebraska--Lincoln}
\vskip-1cm}
%\author{Dr.\ Chris Bourke}
\date{~}
\makeatother

\begin{document}

\maketitle
\vskip-1.5cm
\hrule

\section*{Acceptable Collaboration in CSCE 155E}

\emph{Never} assume that every class or instructor has the same academic 
integrity policies or expectations.  You are expected to understand and 
follow the individual policies of an instructor for every CSCE (and non-CSCE)
course!

All policies follow from the general guideline:  \emph{Any material, technique, 
or resources that makes learning easier is fine.  Any material, technique, 
or resources that makes learning unnecessary is \textbf{not} fine.}

\begin{center}
\begin{tabular}{|l|l|p{8cm}|}
\hline
Labs & Required Collaboration & You are randomly assigned a partner and required to collaborate with them for that lab.  You \emph{may} get help from other groups and
you \emph{should} help other groups but you should primarily work with your assigned partner. \\
\hline
Hacks & Open Collaboration & You \emph{may} work with any number of other students as long as you document whom you collaborated with in your source documentation.\\
\hline
%Assignments & Closed Collaboration & You \emph{may not} work directly with any other student or individual (other than instructors); all work must be your own.  You may only collaborate with other students \emph{at a high level}.  \\
%\hline
Exams & Closed Collaboration & You \emph{may not} work with any other student or individual in any way.  You may not communicate with anyone during the exam.\\
\hline
\end{tabular}
\end{center}

For the online section and for COVID accommodations for fall 2020, 
collaboration is not required for labs but is highly encouraged.

\section*{Acceptable Code Usage}

The following instances are examples of code usage that are acceptable.
Any of the following pieces of code may be used in your labs, hacks, assignments
and exams.

\begin{itemize}
  \item Any function provided as part of a language's standard library
  \item Any piece of code that was provided to you by the instructor
  \item Any piece of code that you wrote as part of a lab, hack session or from another assignment
  \item Any piece of code \emph{you} wrote outside of class or for another class
\end{itemize}

\section*{Unacceptable Code Usage}

The following instances are examples of code usage that are \emph{not}
acceptable.  These instances will be be considered a violation of academic 
integrity for this course.

\begin{itemize}
  \item Any piece of code copied from another student in any form.  This
  includes cut-and-paste of any substantial part, copying entire files, 
  transcribing or reproducing code you saw or copying code and then attempting
  to obfuscate it to hide the copying.
  \item Providing code to another student or allowing another student to copy your code or not taking reasonable measures to prevent another student from gaining access to your code
  \item Making your code publicly accessible prior to final grades posting
  \item Accessing sites such as:
  \begin{itemize}
    \item \url{http://www.chegg.com}
    \item \url{http://coursehero.com}
    \item \url{https://www.freelancer.com/} 
  \end{itemize}
  or anything similar will result in immediate failure of this course.  These
  are sites whose only purpose is to facilitate cheating and have no other 
  legitimate use as far as this course goes.
\end{itemize}

\section*{Don't Be Stupid}

It is \textbf{stupid to cheat} because:
\begin{itemize}
  \item You're not learning anything, you're wasting your money and your time 
  \item Your ignorance will eventually be exposed, either in your exams, future courses, or career; fail fast or fail slow.
\end{itemize}

It is stupid to \textbf{ask for or receive code from someone else} because
\begin{itemize}
  \item If you use it to any significant degree, it will be easily detected using MOSS.
  \item You undermine your own learning experience
  \item If they were willing to share it with you, they likely shared it with others, increasing nearly guaranteeing the likelihood that you will all get caught.
\end{itemize}

It is stupid to \textbf{give your code to someone else} because
\begin{itemize}
  \item They will likely end up screwing you by using it either in part or in whole which will easily be caught using MOSS.  You will suffer the consequences along with them.  You risk everything and gain nothing.
  \item They may share it with any number of other students, almost guaranteeing that you will all get caught.
\end{itemize}

It is stupid to \textbf{copy code off the internet} because
\begin{itemize}
  \item Your instructor can use Google too, you know.  We can just as easily find the source of your plagiarism and in fact can do it easier because we have your copy!
  \item Several other idiots may have the exact same bright idea!  Multiple students copying from the same source is going to be at the top of our red flag list when we run MOSS.
\end{itemize}

\section*{Proper Collaboration}

Though you are allowed to collaborate on labs and hacks, it is important to 
understand what \emph{proper} collaboration is and how to go about engaging
in it.  Collaborating with someone means that you are each putting in an 
\emph{equal effort} and working on every piece of the code together.  
A proper collaboration cannot have a partner who dominates and does all 
the work themselves, undermining the learning experience of the other.
Nor can it be a situation in which one person sits back and \emph{allows}
the other person to do all the work, undermining their own learning 
experience.  The responsibility for ensuring an equal partnership falls
to \emph{each} member.

Collaboration also implies a comparable partnership in terms of skills, 
experience and understanding.  Someone who has had lots of programming 
experience or understands the material naturally should probably not 
``collaborate'' with someone who feels they are struggling with the
topics (and vice versa).  This is an unequal partnership.  It is 
perfectly fine for a more experienced student to \emph{mentor} and 
guide others but in general you should team up with students whom 
you have a more comparable skill/experience level.

Without proper collaboration you end up undermining your own 
learning experience.  It may help you get the labs and hacks done, 
but ultimately you won't learn the material.  When it comes to
the assignments and exams when you have to ``stand on your own''
you will be very ill prepared.  

\newpage
\section*{Academic Integrity Testimonials}

In past semesters when students have been caught cheating or copying code in
a manner that was not allowed by the academic integrity policy, they were asked
to write a testimonial of what they did, why they did it and to give advice
to future students on why they should not make the same mistakes.  These
are their stories.

\begin{quote}
``Computer Science is a course that requires a lot of extra time, it is comparable to learning a new language. As with every course taken from any school, you will get out exactly what you put into it. Cheating or copying code is not the way to learn any new material, not only is an easy escape to class assignment, but will also leave you helpless when it comes to exams and labs, and eventually in the real world. Cheating also shatters any integrity you had, as a syllabus and course contract was signed saying not to cheat, and to be honest with all of the work submitted. As tempting as it is to just copy and paste, even on this one assignment, I highly encourage the reader not to. There are so many alternatives offered by the instructor and college, there is not a good reason to cheat. There are online discussion boards available, extensive office hours from the teachers assistants, graduate teaching assistants, and instructors. A lot of these people are more than willing to help with any questions you may have, and make it very easy to come in and get the help you need. Another great resource is to look back at course slides, notes, and lab work. A lot of this work was demonstrated either in lecture notes, or done in lab with help from fellow classmates and teaching assistants. The instructors and teaching assistants also make it easy to revisit this material.  With all of these options, you are not completely left to your own devices, and there are a fair amount of people who want to help, and there are plenty of resources to look to instead of copying from the internet, or copying off of friends or classmates. In conclusion, I highly suggest putting the time and effort into learning how to use the resources made readily accessible, and not relying on the internet to get assignments done.'' 
\end{quote}

\begin{quote}
``Nothing is worth risking your future and your reputation. You are a good student and a hard worker, but you need to make time and plan ahead for your classes and assignments. Yeah you are overwhelmed with your schedule this year, but there is no excuse for cheating or getting behind. You have plenty of time to re-read lecture notes and practice labs. You need to split focus up equally between classes. It is so pointless to cheat just to get by in a class, because you know deep down you are going to constantly think about the fact you didn't do it on your own, you didn't do it the right way. It will constantly eat at you knowing that you cheated to get the grade you have, when you know if you worked hard enough, you could do even better. And again, if you got caught? You're screwed. You could fail the course or assignment, or even worse. No stupidity or cheating is worth that price of letting your friends and family down. You strive to make your parents proud every day, and this cheating could destroy them and what they think of you. Nothing is worth that. Especially after they have given you all the tools to be successful, do not throw that away. You may be panicking now, but that is because you have not organized your schedule to make time for studying and preparing for every class. This class may get hard at times, but just remember that feeling of gratitude when you finally understand the programming and you start to enjoy coding because you actually know how to use and manipulate code. There was no reason to get behind or late on this assignment, and there was definitely no reason to cheat. You can understand and get an amazing grade by studying and working hard for your grade so you can feel accomplished and deserving of that grade at the end of the semester. You are so much better than this and you can work so much harder and apply that to real life and every other class you are in. Let this message guide you to the right path in being successful and never let cheating be a thought in your mind ever again. Work your hardest and prove to everyone how great you can be with hard work and dedication.''
\end{quote}

\begin{quote}
``In order to help future students with mistakes that may affect their path I have two messages. First, be careful with handing out your code to people. While you may do so with good intentions hoping to help somebody, it may come back around to hurt you. Second, make sure to start all of your homework early. If you are diligent with your work, then you will not be going to the resource center the day homework is due and you will not be seeking other students for help, which might lead to a bad decision.''
\end{quote}

\begin{quote}
``On one of our assignments, I asked a student to share one of his programs with me
because I waited until three days before it was due to start on it. Put simply, I panicked and saw cheating as the only way I could finish the assignment in time. I learned the hard way that when the instructor says they'll catch any cheaters, they most definitely will catch them. In hindsight there were many other options I had, such as starting the assignment while we were covering the material in lecture as opposed to doing it two weeks later. I've never been the type of person to cheat on assignments or tests since I think it's counterproductive and devalues the effort of people who actually did the work. It's wrong for me to get credit for something I put no work into. If I could do it over, I would've just done whatever I could with the time I had and turned it in, even if it wasn't finished.''
\end{quote}

\begin{quote}
``I am writing this paper because I had a program that matched very closely with several other students. The intent was not to cheat, but to instead help someone out. This guy needed some help on a program that I had already completed, and he had helped me out with a problem I was having on another part of the assignment, so I figured why not help him. So, I ended up just sending him my code and he pretty much copied the whole thing word for word.  Not only did he copy it, but he then sent it to more people and they all copied it pretty much word for word. 
	I would highly recommend not to cheat because in the end it's not worth it at all. Eventually, with all the technology that the university has, you will be caught and maybe the professor will be in a bad mood that day and will just kick you out of the class. Then your parents are mad because they just threw money down the drain for that class and now there are even more consequences. So, just struggle through the assignment and actually learn something that you may be able to apply in the future.''
\end{quote}

\begin{quote}
If the thought goes through your mind that this could possible qualify as 
cheating, it more than likely is. Even if it doesn't cross your mind that 
what you're doing could be cheating, it still might be. There was one 
particular assignment that a lot of people were struggling on, so I started 
talking to someone else we knew in the class and just asking things like, 
"Hey, how did you get this to work?" And "Is this how this is supposed to 
be?" Then one thing led to another and I ended up sending another individual 
my code so they could physically look at it and see what was working for us. 
Looking back this was definitely a red flag, but I didn't really think of 
it as that at the time. It is easy and convenient to go to your fellow 
classmates and discuss an assignment and get some help, but what I did 
was a step too far. The University does not take lightly to cheating and 
has a near zero tolerance policy, so just don't do it. Whenever you cheat 
you risk expulsion, and that is not something you want to toy around with.
My advice to you would be to take advantage of the resource center and of 
office hours. I can almost guarantee that they can help you more than anyone 
in the class can. Also, be proactive with going into office hours. If you 
wait until the Friday that the assignment is due you'll spend most of your 
time waiting for a TA to get done helping all the other people that put it 
off before you can get helped.
\end{quote}

\begin{quote}
While Completing an assignment for our Computer Science I class, I turned 
in a program which was almost entirely plagiarized from the internet.  
The most basic reason why this happened was that we did not start the 
assignment in time.  I procrastinated to the point where I started the 
assignment on the day which it was due.  

Plagiarism when it comes to any academic work is not acceptable and 
will be caught.  UNL and every other university takes 
cheating very seriously.  If any person is able to get away with 
cheating, it is not only unfair to the other students in the course, 
but can bring into question the integrity of the program itself.  
You are responsible for upholding the integrity of the University 
and of the College, and if you fail to do so, they will take action 
to ensure that the course remains fair for all students involved and 
that the program remains credible.  Cheating is morally wrong, 
academically dishonest, and hurts not only you, but your classmates, 
the college, and the University.

Even If you are completely lost on an assignment, the professor and TAs 
can and will help you get to a point where you can write a proper 
program, but you need to use your time wisely in order to realize 
that you do not understand something, got to a TA to learn what 
the issue is, and implement a proper fix.  If you put in the proper 
time, you can get an A on every homework assignment.  If you did 
procrastinate, do what you can properly, and accept that your grade 
may not be what you would like.  At that point it is usually too late 
to reach out for help.  Copying code from the internet or from a friend 
will be caught every time, and you will not only receive a worse grade 
than you would have if you had only turned in what you personally could 
write, but you will also face whatever additional punishment the 
professor, college, and university deem necessary.  There is absolutely 
no upside to cheating, and you will not get away with it.  Again, 
if you are in a position where you are not going to be able to complete 
an assignment, accept that you made a mistake, and fix your habits for 
the next assignment.  One bad grade on an assignment will not automatically 
destroy your grade, but, by cheating, you open yourself up to receiving 
absolutely no credit for the assignment, or immediately failing the 
class. Please learn from my mistake and avoid any situation which may 
lead to cheating.
\end{quote}
	
\begin{quote}
To all current CS students or future CS students, always get ahead on 
an assignment and use approved resources to the fullest.  Compared to 
high school, college classes are a lot harder and require a lot more 
work, especially computer science and programming classes.  If you're 
struggling with an assignment and find something online that will work, 
it's not worth it to copy and cheat.  From my experiences, I know that 
it is not worth the risk and consequences to cheat and copy.  It's also 
morally and ethically wrong.  If you get caught cheating, you risk 
getting a 0 on the assignment or an F in the class which compares 
nothing to just trying to figure it out yourself and getting a ``bad'' 
grade on the assignment if you can't end up figuring it out. 
\end{quote}

\begin{quote}
While writing a program, I did my best not to copy a GitHub post directly, 
changing variable names and only taking some of the main elements of the 
program. At the time, it didn't fully register in my mind that I was 
plagiarizing something because I was more focused on having something 
to turn in rather than the quality of what I was turning in. It took 
some time to get the program to compile without any errors and run, 
however it didn't work as it should have. I spent another few hours 
trying to get it to function properly, but couldn't figure it out, 
all the while looking back to the GitHub post more and more to reference 
how it was implemented there, and how I could adapt what I took to work 
in this situation. Ultimately, I never got the program to work

I learned my lesson. I procrastinated too long before starting the assignment and was so caught up in finishing it on time that I resorted to doing something immoral, and I didn't end up with any working programs anyways. Cheating is never worth it.
\end{quote}


\end{document}
