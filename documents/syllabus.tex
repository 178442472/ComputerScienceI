\documentclass[12pt]{scrartcl}


\usepackage{epsfig,amssymb}

\usepackage{xcolor}
\usepackage{graphicx}
\usepackage{epstopdf}
\usepackage{multirow}

\definecolor{darkred}{rgb}{0.5,0,0}
\definecolor{darkgreen}{rgb}{0,0.5,0}
\usepackage{hyperref}
\hypersetup{
  letterpaper,
  colorlinks,
  linkcolor=red,
  citecolor=darkgreen,
  menucolor=darkred,
  urlcolor=blue,
  pdfpagemode=none,
  pdftitle={TITLE},
  pdfauthor={Christopher M. Bourke},
  pdfcreator={$ $Id: cv-us.tex,v 1.28 2009/01/01 00:00:00 cbourke Exp $ $},
  pdfsubject={PhD Thesis},
  pdfkeywords={}
}

\usepackage{fullpage}
\usepackage{tikz}
\pagestyle{empty} %
\usepackage{subfigure}

\definecolor{MyDarkBlue}{rgb}{0,0.08,0.45}
\definecolor{MyDarkRed}{rgb}{0.45,0.08,0}
\definecolor{MyDarkGreen}{rgb}{0.08,0.45,0.08}

\definecolor{mintedBackground}{rgb}{0.95,0.95,0.95}
\definecolor{mintedInlineBackground}{rgb}{.90,.90,1}

\usepackage[newfloat=true]{minted}

\setminted{mathescape,
           linenos,
           autogobble,
           frame=none,
           framesep=2mm,
           framerule=0.4pt,
           %label=foo,
           xleftmargin=2em,
           xrightmargin=0em,
           %startinline=true,  %PHP only, allow it to omit the PHP Tags *** with this option, variables using dollar sign in comments are treated as latex math
           numbersep=10pt, %gap between line numbers and start of line
           style=default} %syntax highlighting style, default is "default"

\setmintedinline{bgcolor={mintedBackground}}
%doesn't work with the above workaround:
\setminted{bgcolor={mintedBackground}}
\setminted[text]{bgcolor={mintedBackground},linenos=false,autogobble,xleftmargin=1em}
%\setminted[php]{bgcolor=mintedBackgroundPHP} %startinline=True}
\SetupFloatingEnvironment{listing}{name=Code Sample}
\SetupFloatingEnvironment{listing}{listname=List of Code Samples}

\setlength{\parindent}{0pt} %
\setlength{\parskip}{.25cm}
\newcommand{\comment}[1]{}

\usepackage{amsmath}
\usepackage{algorithm2e}
\SetKwInOut{Input}{input}
\SetKwInOut{Output}{output}
%NOTE: you can embed algorithms in solutions, but they cannot be floating objects; use [H] to make them non-floats

\usepackage{lastpage}

%\usepackage{titling}
\usepackage{fancyhdr}
\renewcommand*{\titlepagestyle}{fancy}
\pagestyle{fancy}
%\renewcommand*{\titlepagestyle}{fancy}
%\fancyhf{}
%\rhead{Computer Science I}
%\lhead{Guides and tutorials}
\renewcommand{\headrulewidth}{0.0pt}
\renewcommand{\footrulewidth}{0.4pt}

\lhead{~}
\chead{~}
\rhead{~}
\lfoot{\Title\ -- Syllabus}
\cfoot{~}
\rfoot{\thepage\ / \pageref*{LastPage}}

\makeatletter
\title{Computer Science I}\let\Title\@title
\subtitle{Syllabus\\
{\small
\vskip1cm
Department of Computer Science \& Engineering \\
University of Nebraska--Lincoln}
\vskip-1cm}
%\author{Dr.\ Chris Bourke}
\date{Fall 2018}
\makeatother

\begin{document}

\maketitle

\hrule

\begin{quote}
``If you really want to understand something, the best way is to try and explain it to someone else. That forces you to sort it out in your own mind... that's really the essence of programming. By the time you've sorted out a complicated idea into little steps that even a stupid machine can deal with, you've certainly learned something about it yourself.'' 

\hfill ---Douglas Adams, \emph{Dirk Gently's Holistic Detective Agency}
\end{quote}

\begin{quote}
In my experience, you assert control over a computer--show it who's the boss--by making it do something unique. That means programming it... If you devote a couple of hours to programming a new machine, you'll feel better about it ever afterwards'' 

\hfill ---Michael Crichton, Electronic Life
\end{quote}

\section{Course Info}

\begin{tabular}{lp{10cm}}
Prerequisites & Math 103 or equivalent \\
Description  & Introduction to problem solving with computers. 
Topics include problem solving methods, software development 
principles, computer programming, and computing in society.\\
Credit Hours & 3\\
\end{tabular}

For all other information, see the course website.

\section{Skills Objectives}

This course has several learning objectives and ``skills objectives.''
These are the kills that, upon successful completion of this course, 
you should be able to exhibit.

\begin{itemize}
  \item You should have a mastery of the fundamentals of programming 
  in a high-level language, including data types and rudimentary data 
  structures, control flow, repetition, selection, input/output, 
  and procedures and functions.
  \item You should be able to approach a reasonably complex problem, 
  design a top-down solution, and code a program in a high-level programming 
  language that automates solutions.
  \item You should have a familiarity with problem solving methods, 
  including problem analysis, requirements and specifications, design, 
  decomposition and step-wise refinement, and algorithm development 
  (including recursion).
  \item You should have a familiarity with software development principles 
  and practices, including data and operation abstraction, encapsulation, 
  modularity, code and artifact reuse, prototyping, iterative development, 
  best practices in coding design, style, and documentation, a good understanding
  of proper testing and debugging techniques and a familiarity with
  development tools.
  \item You should have exposure to algorithms for searching, sorting 
  and other problems, graphical user interfaces, event-driven programming, 
  and database access. 
  \item You should have a foundation for further software development and
  exploration.  You should have a deep enough understanding of at least
  one high-level programming language that you should be able to learn another
  programming language with relative ease in a relatively short amount of time.  
\end{itemize}

\section{Schedule}

See the course website.

\section{Relationship of Course to ACE}

This course will satisfy Learning Outcome 3: Use computational 
and formal reasoning (including reasoning based on principles 
of logic) to solve problems, draw inferences, and determine 
reasonableness.

\subsection*{Learning Opportunities}
  
The lectures, together with homework and programming assignments and the weekly structured laboratory sessions, teach students methods for developing and implementing algorithms to solve problems.  That is, the course not only teaches students about how to design algorithmic solutions, but also teaches students about how to engineer designs into working software.  The engineering process of designing and implementing a program involves significant debugging, testing, and refining code.  These activities teach and reinforce reasoning and inferencing: a student must develop tests to reasonably indicate program correctness and must draw inferences when diagnosing why a program does not compile, crashes, or generates incorrect output.  Also, an algorithm is fundamentally a logical sequence of steps that, given a set of inputs, generates a set of outputs.
The course includes approximately:
  
\begin{itemize}
  \item 45 hours of lectures each designed to explore concepts and paradigms that are central to the field of computer science.
  \item (at least) 15 hours of structured laboratory sessions, each designed to train students to apply what they learn in the lectures to actual implementations and analyses of algorithms and software.
  \item Several homework and programming assignments designed to help students learn about methods for designing algorithmic solutions and the practices of implementing solutions as correct software.
\end{itemize}
  
\subsection*{Outcome Assessment}
  
A variety of student work is used to assess achievement of the outcomes, including exams, homework and programming assignments, and structured laboratory work.  Exams require students to demonstrate their knowledge in a written format.  The programming assignments are inherently practical demonstrations of problem solving and algorithm development, with reasoning and inferencing to produce programs that compile, run, and compute the correct output. The laboratory work supplements the lectures and to provide supervised hands-on experiences of problem solving, algorithm development, and the realization of a computer solution.  The students submit their results from solving the lab problems and performing the specified tasks.  Laboratory pre-tests, worksheets, and post-tests are graded.  The student must pass pre-tests prior to beginning the lab, complete worksheets with results, and take post-tests prior to the end of the lab.   In summary, all of the following provide opportunities for students to demonstrate their skills related to the learning objective:
\begin{itemize}
  \item There are midterm exams and a comprehensive final exam.   
  \item There are several programming assignments.  
  \item There are weekly structured laboratory assignments.  
\end{itemize}

\section{Accommodations for Students with Disabilities}

It is the policy of the University of Nebraska-Lincoln to 
provide flexible and individualized accommodations to students 
with documented disabilities that may affect their ability to 
fully participate in course activities or to meet course 
requirements.  To receive accommodation services, students 
must be registered with the Services for Students with 
Disabilities (SSD) office, 232 Canfield Administration, 
472-3787 voice or TTY.

%\section{Course Structure}
%
%TODO
%-callback to the quotes above
%-learning opportunities: text, video, handouts, resources
%-expectations on preparing

\section{Grading}

Grading will be based on labs, ``hacks'', assignments, and exams
with the following weighted contribution of each category.

\begin{table}[h]
\centering
\begin{tabular}{lc}
Category & Contribution \\
\hline\hline
Labs  & 15\% \\
Hacks & 30\% \\
Assignments & 30\% \\
Midterm & 10\% \\
Final & 15\% \\
\end{tabular}
\end{table}

\subsection{Scale}

Final letter grades will be awarded based on the following 
standard scale. This scale may be adjusted upwards if the 
instructor deems it necessary based on the final grades only.  
No scale will be made for individual assignments or exams.

\begin{table}[h]
\centering
\begin{tabular}{p{1cm}c}
Letter Grade & Percent \\
\hline\hline
A+ & $\geq 97$ \\
A  & $\geq 93$ \\
A- & $\geq 90$ \\
B+ & $\geq 87$ \\
B  & $\geq 83$ \\
B- & $\geq 80$ \\
C+ & $\geq 77$ \\
C  & $\geq 73$ \\
C- & $\geq 70$ \\
D+ & $\geq 67$ \\
D  & $\geq 63$ \\
D- & $\geq 60$ \\
F  & $<60$ \\
\end{tabular}
\end{table}

\subsection{Labs}

There will be weekly labs that give you hands-on exercises for 
topics recently covered in lecture.  The purpose of lab is not 
only to give you further working experience with lecture topics, 
but also to provide you with additional information and details 
not necessarily covered in lecture.  Each lab will have some 
programming requirements and a supplemental worksheet.  

Labs are setup as a \emph{peer programming} experience.  In each 
lab, you will be randomly paired with a partner.  One of you will 
be the \emph{driver} and the other will be the \emph{navigator}.  
The navigator will be responsible for reading the instructions 
and guiding the driver.  The driver will be in charge of the 
keyboard and will type the code.  Both driver and navigator are 
responsible for developing and working through solutions together.
Neither the navigator nor the driver is ``in charge,'' it is an 
equal partnership.  Beyond your immediate pairing, you are 
encouraged to help and interact and with other pairs in the lab.

Unless otherwise stated, you are required to finish the lab by 
the end of your regular lab meeting time.  A lab instructor must 
sign off on your lab worksheet and you must turn it in to receive 
credit.  Labs that have not been completed on time may not receive
credit. 

\subsection{Hacks}

There will be weekly \emph{hack sessions} that will provide an
opportunity to start working on exercises in an open, collaborative
environment.  Each hack session is a simple program or exercise.
You may not necessarily complete the entire exercise during the hack
session, but it is due by 23:59:59 on Friday in the week in which it
is assigned.

Further details are provided in the handouts, but you are highly 
encouraged to collaborate with any individual and to receive as much
help as you desire on the exercises.  They are intended to ``jumpstart''
you on small programming exercises to give you a good start on the
full programming assignments.

\subsection{Assignments}

There will be several programming assignments that you will work outside
of class/lab.  Carefully read each handout and follow all instructions.
Failure to do so may result in points being deducted.  Full rubrics are
made available through the course website.  

\subsection{Exams}

There will be a midterm exam and a comprehensive final exam.  These
will be open-book, open-note, \emph{required computer} exams.  You
will be doing live coding exercises and submitting them online for
grading.  A working laptop with the proper development tools and
resources is required to take these exams.  If you do not have a 
working laptop computer during the exam dates, please contact us for
alternative accommodations.  More details will be announced closer 
to the exam dates.

\subsection{15th Week Policy Notification}
\label{subsection:deadweek}

A per UNL's 15th Week Policy (available here: \url{https://registrar.unl.edu/academic-standards/policies/fifteenth-week-policy/}) we are required
to serve written notice that the final assignment
as well as the final lab and hack will be due during the 15th 
week or ``dead week.''

\subsection{Grading Policy}

If you have questions about grading or believe that points were 
deducted unfairly, you must first address the issue with the 
individual who graded it to see if it can be resolved.  Such 
questions should be made within a reasonable amount of time 
after the graded assignment has been returned.  No further 
consideration will be given to any assignment a week after 
it grades have been posted.  It is important to emphasize that 
the goal of grading is consistency.  A grade on any given 
assignment, even if it is low for the entire class, should 
not matter that much.  Rather, students who do comparable 
work should receive comparable grades (see the subsection 
on the scale used for this course).

\subsection{Late Work Policy}

In general, there will be no make-up exams.  Exceptions may 
be made in certain circumstances such as health or emergency, 
but you must make every effort to get prior permission.  
Documentation may also be required.

Homework assignments have a strict due date/time as defined by
the CSE server's system clock.  All program files must be handed
in using CSE's webhandin as specified in individual assignment
handouts.  Programs that are even a few seconds past the due 
date/time will be considered late.  

Furthermore, many assignments will have requirements (file 
naming conventions, package requirements, command line input 
requirements, etc.) that will facilitate grading through an 
automated script.  This script has been made available to you 
through the webgrader interface.  You are expected to utilize 
this webgrader interface to ensure that your program is running 
as required and to fix any issues prior to the final due date 
(you may handin and run the script as many times as you like 
up to the due date).  Note, however, that the grader should 
not substitute for developing your own test cases and should 
not be used as the primary resource to debug your program; 
instead it is intended as a last-check mechanism.

It is understandable that unforeseen events may interfere 
with your ability to submit all homework assignments on time.  
As such, this course allows the following late work policy: 
you may hand in any one assignment up to one (academic) week 
late.  Any submissions after a week will not be considered 
and will be given an automatic zero.  Any late submissions 
after using your one \emph{free pass} will not be considered.

You \textbf{may not}, however, use your one late pass on any
lab or hack.  You \textbf{may not} use your one late pass on 
neither the first nor the last assignment.  You may not use it
on the first assignment because doing so is a huge red flag that
you are not approaching this course with the necessary 
self-motivation and preparation necessary.  You may not use
it on the last assignment as it is due during the final week
(see Section \ref{subsection:deadweek}) and we cannot extend
it into the final exam week.

%TODO: consider partners or not
%If you work with a partner on a late assignment, both of your 
%late passes will be forfeit.  If two people work together on an 
%assignment and one of them has already used their late pass, 
%the other may not use their late pass for both of them. 

Failure to adhere to the requirements of an assignment in such 
a manner that makes it impossible to grade your program via 
the webgrader means that a disproportionate amount of time 
would be spent evaluating your assignment.  For this reason, 
we will not grade any assignment that does not compile and 
run through the webgrader.  You will be expected to use your 
late pass to fix the issue(s) before we will consider it for 
grading.  Failure to address the issue or submitting assignments 
with such problems will result in an automatic zero.

\subsection{Academic Integrity}

All homework assignments, programs, and exams must represent
your own work unless otherwise stated.  No collaboration with 
fellow students, past or current, is allowed unless otherwise 
permitted on specific assignments or problems.  The Department of
Computer Science \& Engineering has an Academic Integrity Policy.  
All students enrolled in any computer science course are bound 
by this policy.  You are expected to read, understand, and follow 
this policy.  Violations will be dealt with on a case by case 
basis and may result in a failing assignment or a failing grade 
for the course itself.  The most recent version of the Academic 
Integrity Policy can be found at \url{http://cse.unl.edu/academic-integrity}

\section{Communication \& Getting Help}

The primary means of communication for this course is Piazza, an online
forum system designed for college courses.  We have established a Piazza 
group for this course and you should have received an invitation to join.
If you have not, contact the instructor immediately.  With Piazza you 
can ask questions anonymously, remain anonymous to your classmates, or 
choose to be identified.  Using this open forum system the entire class 
benefits from the instructor and TA responses.  In addition, you and 
other students can also answer each other's questions (again you may
choose to remain anonymous or identify yourself to the instructors or
everyone).  You may still email the instructor or TAs, but more than 
likely you will be redirected to Piazza for help.

In addition, there are two anonymous suggestion boxes that you may 
use to voice your concerns about any problems in the course if you 
do not wish to be identified.  My personal box is available on the
course webpage.  The department also maintains an anonymous suggestion
box available at \url{https://cse.unl.edu/contact-form}.

\subsection{Getting Help}

Your success in this course is ultimately your responsibility.  Your
success in this course depends on how well you utilize the opportunities
and resources that we provide.  There are numerous outlets for learning
the material and getting help in this course:
\begin{itemize}
  \item Lectures: attend lectures regularly and when you do use the 
  time appropriately.  Do not distract yourself with social media or other
  time wasters.  Actively take notes (electronic or hand written).  It is
  well-documented that good note taking directly leads to understanding and
  retention of concepts.
  \item Lecture Videos: Lecture videos are intended as a supplement
  that mirrors lecture material but that may not cover everything.  Watch
  them at your own pace on a regular basis for reiteration or in case
  you missed something in lecture.  
  \item Required Reading: do the required reading on a regular basis.  The
  readings provide additional details and depth that you may not necessarily
  get directly in lecture.  
  \item Labs \& Hack Sessions: use your time during lab and hack sessions 
  wisely.  Engage with your lab instructors, teaching assistants, your partner(s)
  and other students.  Be sure to adequately prepare for labs by reading
  the handouts before coming to lab.  Get started and don't get distracted.
  \item Piazza: if you have questions ask them on Piazza.  It is the best and
  likely fastest way to get help with your questions.  Also, be sure to read
  other student's posts and questions and feel free to answer yourself!
  \item Office Hours \& Student Resource Center: the instructor and teaching
  assistants hold regular office hours throughout the week as posted on the
  course website.  Attend office hours if you have questions or want to 
  review material.  The Student Resource Center (SRC, \url{http://cse.unl.edu/src})
  is open 9AM to 7PM Monday through Friday.  Even if your TAs are not scheduled
  during that time, there are plenty of other TAs and students present that
  may be able to help.  And, you may be able to help others!
  \item Don't procrastinate.  The biggest reason students fail this course
  is because they do not give themselves enough opportunities to learn the
  material.  Don't wait to the last minute to start your assignments.  Many
  people wait to the last minute and flood the TAs and SRC, making it difficult
  to get help as the due date approaches.  Don't underestimate how much time 
  your assignment(s) will take and don't wait to the week before hand to get 
  started.  Ideally, you should be working on the problems as we are covering 
  them.
  \item Get help in the \emph{right way}: when you go to the instructor or
  TA for help, you must demonstrate that you have put forth a good faith 
  effort toward understanding the material.  Asking questions that clearly 
  indicate you have failed to read the required material, have not been
  attending lecture, etc.\ is \emph{not acceptable}.  Don't ask generic
  questions like ``I'm lost, I don't know what I'm doing''.  Instead, 
  explain what you have tried so far.  Explain why you think what you 
  have tried doesn't seem to be working.  Then the TA will have an 
  easier time to help you identify misconceptions or problems.  This 
  is known as ``Rubber Duck Debugging'' where in if you try to explain 
  a problem to someone (or, lacking a live person, a rubber duck), 
  then you can usually identify the problem yourself.  Or, at the very 
  least, get some insight as to what might be wrong.
\end{itemize}



\end{document}
