\documentclass[12pt]{scrartcl}


\usepackage{epsfig,amssymb}

\usepackage{xcolor}
\usepackage{graphicx}
\usepackage{epstopdf}
\usepackage{multirow}

\definecolor{darkred}{rgb}{0.5,0,0}
\definecolor{darkgreen}{rgb}{0,0.5,0}
\usepackage{hyperref}
\hypersetup{
  letterpaper,
  colorlinks,
  linkcolor=red,
  citecolor=darkgreen,
  menucolor=darkred,
  urlcolor=blue,
  bookmarks=true,
  pdfpagemode=none,
  pdftitle={BYOD: Bring Your Own Device},
  pdflang={en},
  pdfauthor={Christopher M. Bourke},
  pdfcreator={$ $Id: cv-us.tex,v 1.28 2009/01/01 00:00:00 cbourke Exp $ $},
  pdfsubject={PhD Thesis},
  pdfkeywords={}
}

\usepackage{fullpage}
\usepackage{tikz}
\pagestyle{empty} %
\usepackage{subfigure}

\definecolor{MyDarkBlue}{rgb}{0,0.08,0.45}
\definecolor{MyDarkRed}{rgb}{0.45,0.08,0}
\definecolor{MyDarkGreen}{rgb}{0.08,0.45,0.08}

\definecolor{mintedBackground}{rgb}{0.95,0.95,0.95}
\definecolor{mintedInlineBackground}{rgb}{.90,.90,1}

\usepackage[newfloat=true]{minted}

\setminted{mathescape,
           linenos,
           autogobble,
           frame=none,
           framesep=2mm,
           framerule=0.4pt,
           %label=foo,
           xleftmargin=2em,
           xrightmargin=0em,
           %startinline=true,  %PHP only, allow it to omit the PHP Tags *** with this option, variables using dollar sign in comments are treated as latex math
           numbersep=10pt, %gap between line numbers and start of line
           style=default} %syntax highlighting style, default is "default"

\setmintedinline{bgcolor={mintedBackground}}
%doesn't work with the above workaround:
\setminted{bgcolor={mintedBackground}}
\setminted[text]{bgcolor={mintedBackground},linenos=false,autogobble,xleftmargin=1em}
%\setminted[php]{bgcolor=mintedBackgroundPHP} %startinline=True}
\SetupFloatingEnvironment{listing}{name=Code Sample}
\SetupFloatingEnvironment{listing}{listname=List of Code Samples}

\setlength{\parindent}{0pt} %
\setlength{\parskip}{.25cm}
\newcommand{\comment}[1]{}

\usepackage{amsmath}
\usepackage{algorithm2e}
\SetKwInOut{Input}{input}
\SetKwInOut{Output}{output}
%NOTE: you can embed algorithms in solutions, but they cannot be floating objects; use [H] to make them non-floats

\usepackage{lastpage}

%\usepackage{titling}
\usepackage{fancyhdr}
\renewcommand*{\titlepagestyle}{fancy}
\pagestyle{fancy}
%\renewcommand*{\titlepagestyle}{fancy}
%\fancyhf{}
%\rhead{Computer Science I}
%\lhead{Guides and tutorials}
\renewcommand{\headrulewidth}{0.0pt}
\renewcommand{\footrulewidth}{0.4pt}
\lfoot{\Title\ -- Computer Science I}
\cfoot{~}
\rfoot{\thepage\ / \pageref*{LastPage}}

\makeatletter
\title{BYOD: Bring Your Own Device}\let\Title\@title
\subtitle{Developing C on Your Own Machine \\
Computer Science I\\
{\small
\vskip.5cm
Department of Computer Science \& Engineering \\
University of Nebraska--Lincoln}
\vskip-2cm}
%\author{Dr.\ Chris Bourke}
\date{~}
\makeatother

\begin{document}

\maketitle

\hrule

\section*{Introduction}

Using the lab computers will always be an option.  However, you may want to
untether yourself from the lab and develop on your own machine, either desktop
or laptop.  There are many options for doing this, the following are only
a few suggestions.

Both of these suggestions have you make a connection to your ``Z-drive'' 
on the CSE server.  This is a remote file system where you can store all
of your files, but interact with them as if they were on your computer.
It is similar to plugging in a memory stick: it will appear as yet another
drive on your computer that you can drag-drop files from/to and open/save
files on.  

These suggestions still have you compiling and running on the CSE server
from the command line.  This saves you from having to install your own C 
compiler/development software.  Also, there are many libraries that your
labs and exercises may require.  All of these have been installed for you
already on the CSE server.  If you choose to go your own way and install
alternative software, you may have difficulty installing libraries if you
don't know what you're doing.

\subsection*{Windows}

\begin{enumerate} 
  \item Connect to your Z: drive using the following FAQ: \\
  \url{https://cse.unl.edu/faq-section/windows#node-290}\\
  This only works on campus; off campus you'd need to setup a VPN.
  \item Install Atom IO \url{https://atom.io/}, a free code/text 
  editor.  Use this editor to edit files on your Z drive once connected.
  
  \textbf{Note}: once installed, you should change the default line ending
  for atom.io.  To do this:
  \begin{enumerate}
    \item Click \mintinline{text}{File} $\rightarrow$ \mintinline{text}{Settings}
    \item Click \mintinline{text}{Packages}
    \item ``Search'' for \mintinline{text}{line-ending-selector}
    \item Under Settings, select \mintinline{text}{LF} instead of \mintinline{text}{CRLF}
  \end{enumerate}
  
  \item Download and install Putty, a ``Secure Shell Client'': \\
  \url{http://www.putty.org/}\\
  Once you've written your program(s), login to the CSE server
  using Putty and compile and run your programs on the CSE server.
\end{enumerate}

\subsection*{Mac}

\begin{enumerate}
  \item Connect to your Z: drive using the following FAQ: \\
  \url{https://cse.unl.edu/faq-section/macintosh#node-295}\\
  This only works on campus; off campus you'd need to setup a VPN.
  \item Install Atom IO \url{https://atom.io/}, a free code/text 
  editor.  Use this editor to edit files on your Z drive once connected.
  \item Use Terminal (Apps $\rightarrow$ Utilities $\rightarrow$ Terminal) to
  login to the CSE server and compile and run on the CSE server.  To login, type
  \mintinline{text}{ssh login@cse.unl.edu} where \mintinline{text}{login} 
  is replaced with your CSE login.
\end{enumerate}

\section*{Getting Started}

\begin{enumerate}
  \item In atom, create a new file named \mintinline{text}{hello.c} and be sure to 
save it on your Z: drive after you've connected.  You can also create directories
on your Z: drive to better organize your code.  
  \item Edit the new file to have the following contents.
\begin{minted}{c}
#include <stdlib.h>
#include <stdio.h>

int main(int argc, char **argv) {

  printf("Hello World!\n");
  return 0;
}
\end{minted}
  \item Using putty (windows) or terminal (mac), compile your file using:\\
  \mintinline{text}{gcc hello.c}
  \item Run your program:\\
  \mintinline{text}{./a.out}
\end{enumerate}

\section*{Virtual Private Network}

In order to access your Z drive from off campus, you'll need to
use a VPN (Virtual Private Network).  UNL's VPN service (and the
required software) can be found here: \url{https://its.unl.edu/services/vpn/}

\section*{Alternatives}

The following are some alternatives if you want to explore further.
All of them are cross-platform solutions.  Keep in mind that you will
need to install and troubleshoot various required libraries yourself if
you choose these.
\begin{itemize} 
  \item Eclipse for C/C++ is a full Integrated Development Environment (IDE):
  \url{https://www.eclipse.org/downloads/eclipse-packages/}
  \item Code::Blocks is another IDE for for C/C++: \url{http://www.codeblocks.org/}
  \item Sublime Text is an excellent code editor but is commercial: \url{https://www.sublimetext.com/}
  \item If you are off campus and need to transfer files between your 
  own machine and the CSE server, you can use an SFTP (Secure File Transfer
  Protocol) client.  We recommend FileZilla: \url{https://filezilla-project.org/}
  (download the \emph{client}, not the server).  For the host, use 
  \mintinline{text}{sftp://cse.unl.edu} and use your cse login and password.
\end{itemize}


  


\end{document}
