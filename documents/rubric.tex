\documentclass[12pt]{scrartcl}

\usepackage[printwatermark,disablegeometry]{xwatermark}

\usepackage{epsfig,amssymb}

\usepackage{xcolor}
\usepackage{graphicx}
\usepackage{epstopdf}
\usepackage{multirow}

\definecolor{darkred}{rgb}{0.5,0,0}
\definecolor{darkgreen}{rgb}{0,0.5,0}
\usepackage{hyperref}
\hypersetup{
  letterpaper,
  colorlinks,
  linkcolor=red,
  citecolor=darkgreen,
  menucolor=darkred,
  urlcolor=blue,
  pdfpagemode=none,
  pdftitle={TITLE},
  pdfauthor={Christopher M. Bourke},
  pdfcreator={$ $Id: cv-us.tex,v 1.28 2009/01/01 00:00:00 cbourke Exp $ $},
  pdfsubject={PhD Thesis},
  pdfkeywords={}
}

\usepackage{fullpage}
\usepackage{tikz}
\pagestyle{empty} %
\usepackage{subfigure}

\definecolor{MyDarkBlue}{rgb}{0,0.08,0.45}
\definecolor{MyDarkRed}{rgb}{0.45,0.08,0}
\definecolor{MyDarkGreen}{rgb}{0.08,0.45,0.08}

\definecolor{mintedBackground}{rgb}{0.95,0.95,0.95}
\definecolor{mintedInlineBackground}{rgb}{.90,.90,1}

\usepackage[newfloat=true]{minted}

\setminted{mathescape,
           linenos,
           autogobble,
           frame=none,
           framesep=2mm,
           framerule=0.4pt,
           %label=foo,
           xleftmargin=2em,
           xrightmargin=0em,
           %startinline=true,  %PHP only, allow it to omit the PHP Tags *** with this option, variables using dollar sign in comments are treated as latex math
           numbersep=10pt, %gap between line numbers and start of line
           style=default} %syntax highlighting style, default is "default"

\setmintedinline{bgcolor={mintedBackground}}
%doesn't work with the above workaround:
\setminted{bgcolor={mintedBackground}}
\setminted[text]{bgcolor={mintedBackground},linenos=false,autogobble,xleftmargin=1em}
%\setminted[php]{bgcolor=mintedBackgroundPHP} %startinline=True}
\SetupFloatingEnvironment{listing}{name=Code Sample}
\SetupFloatingEnvironment{listing}{listname=List of Code Samples}

\setlength{\parindent}{0pt} %
\setlength{\parskip}{.25cm}
\newcommand{\comment}[1]{}

\usepackage{amsmath}
\usepackage{algorithm2e}
\SetKwInOut{Input}{input}
\SetKwInOut{Output}{output}
%NOTE: you can embed algorithms in solutions, but they cannot be floating objects; use [H] to make them non-floats

\usepackage{lastpage}

%\usepackage{titling}
\usepackage{fancyhdr}
\renewcommand*{\titlepagestyle}{fancy}
\pagestyle{fancy}
%\renewcommand*{\titlepagestyle}{fancy}
%\fancyhf{}
%\rhead{Computer Science I}
%\lhead{Guides and tutorials}
\renewcommand{\headrulewidth}{0.0pt}
\renewcommand{\footrulewidth}{0.4pt}

\lhead{~}
\chead{~}
\rhead{~}
\lfoot{\Title\ -- Code Rubric}
\cfoot{~}
\rfoot{\thepage\ / \pageref*{LastPage}}

\makeatletter
\title{Computer Science I}\let\Title\@title
\subtitle{Code Rubric\\
{\small
\vskip1cm
Department of Computer Science \& Engineering \\
University of Nebraska--Lincoln}
\vskip-1cm}
%\author{Dr.\ Chris Bourke}
\date{Fall 2018}
\makeatother

\begin{document}

\maketitle

%\newwatermark[allpages=true,scale=5,textmark=Draft]{},

\hrule

This is a draft of items/elements that will be included in rubrics for
assignments and hacks.

\subsection*{Following Instructions}
\begin{itemize}
  \item All required soft-copy files handed in via webhandin
  \item Correct file name(s) and organization
  \item Programs successfully compile and execute using the webgrader
\end{itemize}

\subsection*{Style}
\begin{itemize}
  \item Appropriate variable and function/method identifiers
  \item Style and naming conventions are consistent
  \item Good use of whitespace; proper indentation
  \item Clean, readable code
  \item Code is well-organized
\end{itemize}

\subsection*{Documentation}
\begin{itemize}
  \item Well written comments that clearly explain the purpose of each non-trivial piece of code
  \item Comments explain the ``what'' and ``why''
  \item Comments are not overly verbose or overly terse
  \item Code itself is ``self-documenting''; it explains the ``how''
\end{itemize}

\subsection*{Program Design}
\begin{itemize}
  \item Code is well-organized and efficient
  \item Code is modular; substantial pieces of it could be reused; few redundancies
  \item Code is easily understood and maintainable
  \item It is clear that sufficient testing has been performed
  \item Corner cases and bad input have been anticipated and appropriate error handling has been implemented  
\end{itemize}

\subsection*{Program Correctness}
\begin{itemize}
  \item Source code compiles and executes as expected
  \item Program runs as specified: correctly reads any input; correctly formatted output
  \item Test cases successfully execute
\end{itemize}

\end{document}
