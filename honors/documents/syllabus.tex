\documentclass[12pt]{scrartcl}

\usepackage[printwatermark,disablegeometry]{xwatermark}

\usepackage{epsfig,amssymb}

\usepackage{xcolor}
\usepackage{graphicx}
\usepackage{epstopdf}
\usepackage{multirow}
\usepackage{colortbl} 


\definecolor{steelblue}{RGB}{70, 130, 180}
\definecolor{darkred}{rgb}{0.5,0,0}
\definecolor{darkgreen}{rgb}{0,0.5,0}
\usepackage{hyperref}
\hypersetup{
  letterpaper,
  colorlinks,
  linkcolor=red,
  citecolor=darkgreen,
  menucolor=darkred,
  urlcolor=blue,
  pdfpagemode=none,
  pdftitle={Syllabus},
  pdfauthor={Christopher M. Bourke},
  pdfkeywords={}
}

\usepackage{fullpage}
\usepackage{tikz}
\pagestyle{empty} %
\usepackage{subfigure}

\definecolor{MyDarkBlue}{rgb}{0,0.08,0.45}
\definecolor{MyDarkRed}{rgb}{0.45,0.08,0}
\definecolor{MyDarkGreen}{rgb}{0.08,0.45,0.08}

\definecolor{mintedBackground}{rgb}{0.95,0.95,0.95}
\definecolor{mintedInlineBackground}{rgb}{.90,.90,1}

\usepackage[newfloat=true]{minted}

\setminted{mathescape,
           linenos,
           autogobble,
           frame=none,
           framesep=2mm,
           framerule=0.4pt,
           %label=foo,
           xleftmargin=2em,
           xrightmargin=0em,
           %startinline=true,  %PHP only, allow it to omit the PHP Tags *** with this option, variables using dollar sign in comments are treated as latex math
           numbersep=10pt, %gap between line numbers and start of line
           style=default} %syntax highlighting style, default is "default"

\setmintedinline{bgcolor={mintedBackground}}
%doesn't work with the above workaround:
\setminted{bgcolor={mintedBackground}}
\setminted[text]{bgcolor={mintedBackground},linenos=false,autogobble,xleftmargin=1em}
%\setminted[php]{bgcolor=mintedBackgroundPHP} %startinline=True}
\SetupFloatingEnvironment{listing}{name=Code Sample}
\SetupFloatingEnvironment{listing}{listname=List of Code Samples}

\setlength{\parindent}{0pt} %
\setlength{\parskip}{.25cm}
\newcommand{\comment}[1]{}

\usepackage{amsmath}
\usepackage{algorithm2e}
\SetKwInOut{Input}{input}
\SetKwInOut{Output}{output}
%NOTE: you can embed algorithms in solutions, but they cannot be floating objects; use [H] to make them non-floats

\usepackage{lastpage}

%\usepackage{titling}
\usepackage{fancyhdr}
\renewcommand*{\titlepagestyle}{fancy}
\pagestyle{fancy}
%\renewcommand*{\titlepagestyle}{fancy}
%\fancyhf{}
%\rhead{Computer Science I}
%\lhead{Guides and tutorials}
\renewcommand{\headrulewidth}{0.0pt}
\renewcommand{\footrulewidth}{0.4pt}

\lhead{~}
\chead{~}
\rhead{~}
\lfoot{\Title\ -- CSCE 155H Syllabus}
\cfoot{~}
\rfoot{\thepage\ / \pageref*{LastPage}}

\makeatletter
\title{Computer Science I}\let\Title\@title
\subtitle{Syllabus\\
{\small
\vskip1cm
Department of Computer Science \& Engineering \\
University of Nebraska--Lincoln}
\vskip-1cm}
%\author{Dr.\ Chris Bourke}
\date{CSCE 155H -- Fall 2020}
\makeatother

\begin{document}

\maketitle

%\newwatermark[allpages=true,scale=5,textmark=Draft]{},

\hrule

\begin{quote}
``If you really want to understand something, the best way is to try and explain it to someone else. That forces you to sort it out in your own mind... that's really the essence of programming. By the time you've sorted out a complicated idea into little steps that even a stupid machine can deal with, you've certainly learned something about it yourself.'' 

\hfill ---Douglas Adams, \emph{Dirk Gently's Holistic Detective Agency}
\end{quote}

\begin{quote}
In my experience, you assert control over a computer--show it who's the boss--by making it do something unique. That means programming it... If you devote a couple of hours to programming a new machine, you'll feel better about it ever afterwards'' 

\hfill ---Michael Crichton, Electronic Life
\end{quote}

\begin{quote}
The readings were the most underrated things ever (I've never actually read them but I wish I did) thanks for providing them and leaving the whole course available to us. 

\hfill ---Former student a year after taking this course
\end{quote}

\begin{quote}
Enjoyed the videos. Did not really watch them videos until [later],
realized that it was a mistake now.

\hfill ---Former student a year after taking this course
\end{quote}

\begin{quote}
``There are only two kinds of languages: the ones people complain 
about and the ones nobody uses.''

\hfill ---Bjarne Stroustrup (creator of C++)
\end{quote}

\begin{quote}
``But be aware that you won't reach the skill level of a hacker or 
even merely a programmer if you only know one language--you need to 
learn how to think about programming problems in a general way, 
independent of any one language.  To be a real hacker, you need to 
have gotten to the point where you can learn a new language in days 
by relating what's in the manual to what you already know.  This 
means you should learn several very different languages.''

\hfill ---Eric S. Raymond, How to Become a Hacker (The Cathedral and the Bazaar)
\end{quote}

\begin{quote}
``I came into this class able to code by constantly using references 
and needing to double check my code. I left as a coding machine. The 
rate at which I did the last homework was like that of whole other 
person.''

\hfill ---Previous student via course evaluation
\end{quote}




\section{Course Info}

\textbf{Prerequisites}: MATH 103 or equivalent.

\textbf{Description}: Introduction to problem solving with computers. 
Topics include problem solving methods, software development 
principles, computer programming, and computing in society.

\textbf{Credit Hours}: 3

\textbf{Postrequisites}: The course after this course, CSCE 156 -- Computer 
Science II requires that you receive a grade of C or better in this course to
move on.  If you are a Computer Science or Computer Engineering major you will
need to receive a C or better in this course to continue in the major.  

For all other information, see the course website.

\section{Skills Objectives}

This course has several learning objectives and ``skills objectives.''
These are the skills that, upon successful completion of this course, 
you should be able to exhibit.

\begin{itemize}
  \item You should have a mastery of the fundamentals of programming 
  in a high-level language, including data types and rudimentary data 
  structures, control flow, repetition, selection, input/output, 
  and procedures and functions.
  \item You should be able to approach a reasonably complex problem, 
  design a top-down solution, and code a program in a high-level programming 
  language that automates solutions.
  \item You should have a familiarity with problem solving methods, 
  including problem analysis, requirements and specifications, design, 
  decomposition and step-wise refinement, and algorithm development 
  (including recursion).
  \item You should have a familiarity with software development principles 
  and practices, including data and operation abstraction, encapsulation, 
  modularity, code and artifact reuse, prototyping, iterative development, 
  best practices in coding design, style, and documentation, a good understanding
  of proper testing and debugging techniques and a familiarity with
  development tools.
  \item You should have exposure to algorithms for searching, sorting 
  and other problems, graphical user interfaces, event-driven programming, 
  and database access. 
  \item You should have a foundation for further software development and
  exploration.  You should have a deep enough understanding of at least
  one high-level programming language that you should be able to learn another
  programming language with relative ease in a relatively short amount of time.  
\end{itemize}

\section{Honors Section}

As an Honors course, topics are covered in a greater depth than the main
section of this course.  We cover the same topics as the main section, 
but we do so by covering two programming languages: C and Java.  This
gives us an opportunity to highlight the differences and idiosyncrasies 
of both languages.  Assignments include exercises in both languages.  
Likewise, exams will also have questions that require knowledge of both 
languages.  All students are responsible for material on both languages.
Weekly lab assignments will also have two versions: one in C and one in 
Java.  Both versions will cover the same concepts and have similar 
exercises.  However, each student is required to complete only one 
version (the choice is left to the student).  However, you are highly 
encouraged to complete both versions to gain a better knowledge of both 
languages.


\section{Schedule}

See the course website.

\section{Accommodations for Students with Disabilities}

%updated from https://www.unl.edu/ssd/content/syllabus-statement-faculty
% 2020/07/01

The University strives to make all learning experiences as 
accessible as possible. If you anticipate or experience 
barriers based on your disability (including mental health, 
chronic or temporary medical conditions), please let me know 
immediately so that we can discuss options privately. To 
establish reasonable accommodations, I may request that you 
register with Services for Students with Disabilities (SSD). 
If you are eligible for services and register with their 
office, make arrangements with me as soon as possible to 
discuss your accommodations so they can be implemented in a 
timely manner. SSD contact information:  117 Louise Pound 
Hall.; 402-472-3787

\section{Grading}

Assessment (grading) will be based assignments, labs, 
and two exams with the following point distributions.

\begin{table}[h]
\centering
{\small
\setlength{\tabcolsep}{0.5em} % for the horizontal padding
\renewcommand{\arraystretch}{1.2}% for the vertical padding

\begin{tabular}{lrrr}
\hline
\rowcolor{steelblue!50} Category & Number & Points Each & Total \\
\hline
\rowcolor{steelblue!5}   Starter Points    &    &     &  10 \\
\rowcolor{steelblue!10}  Attendance        &    &     &   0 \\
\rowcolor{steelblue!5}   Labs              & 14 & 10  & 140 \\
\rowcolor{steelblue!10}  Assignments       &  6 & 100 & 600 \\
\rowcolor{steelblue!5}   Midterm           & 1  & 100 & 100 \\
\rowcolor{steelblue!10}  Final             & 1  & 150 & 150 \\
\hline
Total  & & & 1,000 
\end{tabular}
}
\end{table}


\subsection{Starter Points}

It is important to start out positively.  Put yourself in the mindset
that you \emph{will} succeed in this course and commit yourself to
putting in a full effort in every aspect of it.  To get you started,
we're giving you 10 free points.  You have a perfect score in this
course already!  Keep it up!

\subsection{Attendance \& COVID-19}

\subsubsection{Face Covering}

\textbf{Required Use of Face Coverings for On-Campus Shared Learning Environments}

As of July 17, 2020 and until further notice, all University of 
Nebraska--Lincoln (UNL) faculty, staff, students, and visitors 
(including contractors, service providers, and others) are required 
to use a facial covering at all times when indoors except under 
specific conditions outlined in the COVID 19 face covering policy 
found at: \url{https://covid19.unl.edu/face-covering-policy}. This 
statement is meant to clarify classroom policies for face coverings:

To protect the health and well-being of the University and wider 
community, UNL has implemented a policy requiring all people, 
including students, faculty, and staff, to wear a face covering 
that covers the mouth and nose while on campus. The classroom is 
a community, and as a community, we seek to maintain the health 
and safety of all members by wearing face coverings when in the 
classroom. Failure to comply with this policy is interpreted as 
a disruption of the classroom and may be a violation of UNL's 
Student Code of Conduct.

Individuals who have health or medical reasons for not wearing 
face coverings should work with the Office of Services for 
Students with Disabilities (for students) or the Office of 
Faculty/Staff Disability Services (for faculty and staff) to 
establish accommodations to address the health concern. Students 
who prefer not to wear a face covering should work with their 
advisor to arrange a fully online course schedule that does 
not require their presence on campus.

Students in the classroom:
\begin{enumerate}
  \item If a student is not properly wearing a face covering, 
  the instructor will remind the student of the policy and ask 
  them to comply with it.
  \item If the student will not comply with the face covering policy, 
  the instructor will ask the student to leave the classroom, 
  and the student may only return when they are properly wearing 
  a face covering.
  \item If the student refuses to properly wear a face covering 
  or leave the classroom, the instructor will dismiss the class 
  and will report the student to Student Conduct \& Community 
  Standards for misconduct, where the student will be subject to disciplinary action.
\end{enumerate}

Instructors in the classroom:
\begin{enumerate}
  \item If an instructor is not properly wearing a face covering, 
  students will remind the instructor of the policy and ask them 
  to comply with it.
  \item If an instructor will not properly wear a face covering, 
  students may leave the classroom and should report the misconduct 
  to the department chair or via the TIPS system for disciplinary 
  action through faculty governance processes.
\end{enumerate}

\subsubsection{Social Distancing \& Attendance}

Due to the continuing \textbf{COVID-19 pandemic}, the following 
policies and accommodations will be made for this course.  

\begin{itemize}
  \item No assessment will be made based on any synchronous attendance 
  for instructional sessions (which includes lecture, lab, and 
  office hours).  The choice to attend face-to-face
  instructional sessions is left entirely to your own judgment.  If you have
  a medical condition or other consideration or simply do not wish to
  unnecessarily expose yourself, you are \emph{encouraged} to refrain
  from face-to-face sessions.  You will not be judged nor will you be
  required to provide any documentation.
  
  \item If you feel sick, have a fever or other related symptoms, you
  \emph{may not} attend face-to-face sessions.  Please attend remotely.
  
  \item \textbf{Lecture}: we are fortunate enough that our lecture
  room will accommodate all students enrolled and so you 
  \emph{may} attend lecture face-to-face for every lecture session
  
  \item \textbf{Lab}: Our normal lab meeting
  room has had its capacity reduced to half (Avery 20, capacity 15).  
  However, the department has secured a secondary \emph{overflow} room
  on the same day/time.  This overflow rooms are posted in 
  Canvas but will \emph{not} have lab computers.  You may attend
  on a BYOD (Bring Your Own Device) basis.  If you plan on using 
  your own laptop and attending face-to-face, 
  please make this overflow room your first choice.  Once either
  room is filled to capacity, please go to the other room or 
  consider attending remotely via zoom.
  
\end{itemize}

These are difficult conditions.  Under normal conditions attendance
would generally be required and assessed as there is a 
\emph{strong correlation} with students who regularly 
attend lecture and labs and those who succeed academically
in this course. 

This course will be ever more challenging by having fewer of these
face-to-face opportunities.  However, we hope to make up for it using
virtual sessions (via zoom) and have eliminated some assessments
for this course offering.  


%There is a \emph{strong correlation} with students who regularly 
%attend lecture, labs, and hacks and those who succeed academically
%in this course.  For example, students who regularly attended hack
%sessions in Fall 2019 earned a full letter grade better in their
%hack submissions than those who did not.  Note it is a \emph{correlation}
%simply \emph{being} there is not enough, but it \emph{is} necessary.
%To that end, we require and record attendance at all lectures (29 total), 
%labs (14 total), and hack sessions (15 total).  
%
%Attendance will be recording either using an app (TODO) or by Learning
%Assistants.  It is understandable that you may inevitably miss a class
%or two due to illness or other obligations.  To accommodate such events, 
%attendance only represents 50 points (50 events out of a possible 58).
%Thus, up to 8 events can be missed without impacting your grade.  

%\subsubsection*{What Attendance Means}
%
%Attendance is not merely \emph{being present} but \emph{being engaged}.
%In lecture that means being attentive, paying attention and taking notes.
%Countless studies have shown that the simple act of taking notes improves
%retention and comprehension.  In labs and hacks that means actively 
%engaging in the relevant work, asking for help and actively engaging
%in \emph{collaboration}.  To that end, you may not leave lab
%or hack early or when you have completed your own tasks.  Instead, you 
%are to use the remaining time to assist your peers.  Share your
%knowledge and your skills and do so in a positive and constructive manner.

\subsection{Labs}

There will be weekly labs that give you hands-on exercises for 
topics recently covered in lecture.  The purpose of lab is not 
only to give you further working experience with lecture topics, 
but also to provide you with additional information and details 
not necessarily covered in lecture.  Each lab will have some 
programming requirements and a supplemental worksheet.  

Depending on logistics, those in the on-campus section may be
randomly paired with a partner.  One of you will 
be the \emph{driver} and the other will be the \emph{navigator}.  
The navigator will be responsible for reading the instructions 
and guiding the driver.  The driver will be in charge of the 
keyboard and will type the code.  Both driver and navigator are 
responsible for developing and working through solutions together.
Neither the navigator nor the driver is ``in charge,'' it is an 
equal partnership.  Beyond your immediate pairing, you are 
encouraged to help and interact and with other pairs in the lab.

For those in the online section or those attending remotely, you 
may work alone or we may partner you with someone in a zoom 
breakout room.  

Each lab is assessed based on completion.  In general you will
need to submit code and an electronic writeup of your worksheet
(plain text or markdown is preferred) through webhandin which
can also be verified using the webgrader.  You will have until
midnight on the day of the lab (Tuesdays) to submit your solutions.
Points will be awarded based on the results of the webgrader.


%Labs are setup as a \emph{peer programming} experience.  In each 
%lab, you will be randomly paired with a partner.  One of you will 
%be the \emph{driver} and the other will be the \emph{navigator}.  
%The navigator will be responsible for reading the instructions 
%and guiding the driver.  The driver will be in charge of the 
%keyboard and will type the code.  Both driver and navigator are 
%responsible for developing and working through solutions together.
%Neither the navigator nor the driver is ``in charge,'' it is an 
%equal partnership.  Beyond your immediate pairing, you are 
%encouraged to help and interact and with other pairs in the lab.
%
%Unless otherwise stated, you are required to finish the lab by 
%the end of your regular lab meeting time.  A lab instructor must 
%sign off on your lab worksheet and you must turn it in to receive 
%credit.  Labs not completed on time \emph{may not} receive
%credit. This means that you \emph{must} show up to your lab on
%time and be properly prepared.  Being prepared means having gone
%through the relevant material and pre-lab required reading.  It 
%means not being preoccupied with material or questions not related
%to the lab material.  

\subsection{Assignments}

There are 6 assignments each constituting of several programming
exercises.  Code and other relevant files must be submitted using 
CSE's webhandin.  Many assignments will have requirements (file 
names, package requirements, command line input requirements, 
etc.) that will facilitate grading through an automated script.  
This script has been made available to you through the webgrader 
interface.  You are expected to utilize this webgrader interface 
to ensure that your program is running as required and to fix any 
issues prior to the final due date (you may handin and run the 
script as many times as you like up to the due date).  If your 
program fails to compile or run through the webgrader interface, 
you may receive a zero.

Understand that the webgrader interface is a black box tester.  
It is not a substitute for developing your own test cases and 
should not be used as the primary resource to debug your program; 
instead it is intended as a last-check mechanism.

The final homework may be due as late as Friday of Dead Week.  
As per the 15th week policy, this serves as written notice.

\subsection{Exams}

There will be one midterm exam and a comprehensive final exam.  These
will be open-book, open-note, \emph{required computer} exams.  The
exams consist of live coding exercises for which you will need your
own machine as you will be coding and submitting programs online for
grading.  More details will be announced closer to the exam dates.
Due to attendance limitations, these will likely be ``take home''
exams.  You will be on your own honor to complete these exams alone
and without collaboration.

%There will be one midterm exam and a comprehensive final exam.  These
%will be open-book, open-note, \emph{required computer} exams.  The
%exams consist of live coding exercises for which you will need your
%own machine as you will be coding and submitting programs online for
%grading.  A working laptop with the proper development tools and
%resources is required to take these exams.  If you do not have a 
%working laptop computer during the exam dates, please contact us for
%alternative accommodations.  More details will be announced closer 
%to the exam dates.

\subsection{15th Week Policy Notification}
\label{subsection:deadweek}

A per UNL's 15th Week Policy (also known as ``dead week'') available here: 

\url{https://registrar.unl.edu/academic-standards/policies/fifteenth-week-policy/}

we are required to serve written notice that the final assignment
as well as the final lab, hack, and assignment will be due during the 15th 
week.

\subsection{Scale}

Final letter grades will be awarded based on the following 
standard scale. This scale may be adjusted upwards if the 
instructor deems it necessary based on the final grades only.  
No scale will be made for individual assignments or exams.

\begin{table}[h]
\centering
\begin{tabular}{p{1cm}c}
Letter Grade & Percent \\
\hline\hline
A+ & $\geq 97$ \\
A  & $\geq 93$ \\
A- & $\geq 90$ \\
B+ & $\geq 87$ \\
B  & $\geq 83$ \\
B- & $\geq 80$ \\
C+ & $\geq 77$ \\
C  & $\geq 73$ \\
C- & $\geq 70$ \\
D+ & $\geq 67$ \\
D  & $\geq 63$ \\
D- & $\geq 60$ \\
F  & $<60$ \\
\end{tabular}
\end{table}

\subsection{Grading Policy}

If you have questions about grading or believe that points were 
deducted unfairly, you must first address the issue with the 
individual who graded it to see if it can be resolved.  Such 
questions should be made within a reasonable amount of time 
after the graded assignment has been returned.  No further 
consideration will be given to any assignment a week after 
it grades have been posted.  It is important to emphasize that 
the goal of grading is consistency.  A grade on any given 
assignment, even if it is low for the entire class, should 
not matter that much.  Rather, students who do comparable 
work should receive comparable grades (see the subsection 
on the scale used for this course).

\subsection{Late Work Policy}

In general, there will be no make-up exams or late work
accepted.  Exceptions may be made in certain circumstances 
such as health or emergency, but you must make every effort 
to get prior permission.  Documentation may also be required.

Homework assignments have a strict due date/time as defined by
the CSE server's system clock.  All program files must be handed
in using CSE's webhandin as specified in individual assignment
handouts.  Programs that are even a few seconds past the due 
date/time will be considered late and you will be locked out
of handing anything in after that time.  

\subsection{Webgrader Policy}

Failure to adhere to the requirements of an assignment in such 
a manner that makes it impossible to grade your program via 
the webgrader means that a disproportionate amount of time 
would be spent evaluating your assignment.  For this reason, 
we will not grade any assignment that does not compile and 
run through the webgrader.  

\subsection{Academic Integrity}

All homework assignments, programs, and exams must represent
your own work unless otherwise stated.  No collaboration with 
fellow students, past or current, is allowed unless otherwise 
permitted on specific assignments or problems.  The Department of
Computer Science \& Engineering has an Academic Integrity Policy.  
All students enrolled in any computer science course are bound 
by this policy.  You are expected to read, understand, and follow 
this policy.  Violations will be dealt with on a case by case 
basis and may result in a failing assignment or a failing grade 
for the course itself.  The most recent version of the Academic 
Integrity Policy can be found at \url{http://cse.unl.edu/academic-integrity}

\section{Communication \& Getting Help}

The primary means of communication for this course is Piazza, an online
forum system designed for college courses.  We have established a Piazza 
group for this course and you should have received an invitation to join.
If you have not, contact the instructor immediately.  With Piazza you 
can ask questions anonymously, remain anonymous to your classmates, or 
choose to be identified.  Using this open forum system the entire class 
benefits from the instructor and TA responses.  In addition, you and 
other students can also answer each other's questions (again you may
choose to remain anonymous or identify yourself to the instructors or
everyone).  You may still email the instructor or TAs, but more than 
likely you will be redirected to Piazza for help.

In addition, there are two anonymous suggestion boxes that you may 
use to voice your concerns about any problems in the course if you 
do not wish to be identified.  My personal box is available on the
course webpage.  The department also maintains an anonymous suggestion
box available at \url{https://cse.unl.edu/contact-form}.

\subsection{Getting Help}

Your success in this course is ultimately your responsibility.  Your
success in this course depends on how well you utilize the opportunities
and resources that we provide.  There are numerous outlets for learning
the material and getting help in this course:
\begin{itemize}
  \item Lectures: attend lectures regularly and when you do use the 
  time appropriately.  Do not distract yourself with social media or other
  time wasters.  Actively take notes (electronic or hand written).  It is
  well-documented that good note taking directly leads to understanding and
  retention of concepts.
  \item Lecture Videos: Lecture videos are intended as a supplement
  that mirrors lecture material but that may not cover everything.  Watch
  them at your own pace on a regular basis for reiteration or in case
  you missed something in lecture.  
  \item Required Reading: do the required reading on a regular basis.  The
  readings provide additional details and depth that you may not necessarily
  get directly in lecture.  
  \item Labs \& Hack Sessions: use your time during lab and hack sessions 
  wisely.  Engage with your lab instructors, teaching assistants, your partner(s)
  and other students.  Be sure to adequately prepare for labs by reading
  the handouts before coming to lab.  Get started and don't get distracted.
  \item Piazza: if you have questions ask them on Piazza.  It is the best and
  likely fastest way to get help with your questions.  Also, be sure to read
  other student's posts and questions and feel free to answer yourself!
  \item Office Hours \& Student Resource Center: the instructor and teaching
  assistants hold regular office hours throughout the week as posted on the
  course website.  Attend office hours if you have questions or want to 
  review material.  The Student Resource Center (SRC, \url{http://cse.unl.edu/src})
  Monday through Friday.  Even if your TAs are not scheduled
  during that time, there are plenty of other TAs and students present that
  may be able to help.  And, you may be able to help others!
  \item Don't procrastinate.  The biggest reason students fail this course
  is because they do not give themselves enough opportunities to learn the
  material.  Don't wait to the last minute to start your assignments.  Many
  people wait to the last minute and flood the TAs and SRC, making it difficult
  to get help as the due date approaches.  Don't underestimate how much time 
  your assignment(s) will take and don't wait to the week before hand to get 
  started.  Ideally, you should be working on the problems as we are covering 
  them.
  \item Get help in the \emph{right way}: when you go to the instructor or
  TA for help, you must demonstrate that you have put forth a good faith 
  effort toward understanding the material.  Asking questions that clearly 
  indicate you have failed to read the required material, have not been
  attending lecture, etc.\ is \emph{not acceptable}.  Don't ask generic
  questions like ``I'm lost, I don't know what I'm doing''.  Instead, 
  explain what you have tried so far.  Explain why you think what you 
  have tried doesn't seem to be working.  Then the TA will have an 
  easier time to help you identify misconceptions or problems.  This 
  is known as ``Rubber Duck Debugging'' where in if you try to explain 
  a problem to someone (or, lacking a live person, a rubber duck), 
  then you can usually identify the problem yourself.  Or, at the very 
  least, get some insight as to what might be wrong.
\end{itemize}



\end{document}
