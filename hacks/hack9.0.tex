\documentclass[12pt]{scrartcl}


\usepackage{epsfig,amssymb}

\usepackage{xcolor}
\usepackage{graphicx}
\usepackage{epstopdf}
\usepackage{multirow}

\definecolor{darkred}{rgb}{0.5,0,0}
\definecolor{darkgreen}{rgb}{0,0.5,0}
\usepackage[pdfusetitle]{hyperref}
\hypersetup{
  letterpaper,
  colorlinks,
  linkcolor=red,
  citecolor=darkgreen,
  menucolor=darkred,
  urlcolor=blue,
  pdfpagemode=none,
}

\usepackage{fullpage}
\usepackage{tikz}
\pagestyle{empty} %
\usepackage{subfigure}

\definecolor{MyDarkBlue}{rgb}{0,0.08,0.45}
\definecolor{MyDarkRed}{rgb}{0.45,0.08,0}
\definecolor{MyDarkGreen}{rgb}{0.08,0.45,0.08}

\definecolor{mintedBackground}{rgb}{0.95,0.95,0.95}
\definecolor{mintedInlineBackground}{rgb}{.90,.90,1}

\usepackage[newfloat=true]{minted}

\setminted{mathescape,
           linenos,
           autogobble,
           frame=none,
           framesep=2mm,
           framerule=0.4pt,
           %label=foo,
           xleftmargin=2em,
           xrightmargin=0em,
           %startinline=true,  %PHP only, allow it to omit the PHP Tags *** with this option, variables using dollar sign in comments are treated as latex math
           numbersep=10pt, %gap between line numbers and start of line
           style=default} %syntax highlighting style, default is "default"

\setmintedinline{bgcolor={mintedBackground}}
%doesn't work with the above workaround:
\setminted{bgcolor={mintedBackground}}
\setminted[text]{bgcolor={mintedBackground},linenos=false,autogobble,xleftmargin=1em}
%\setminted[php]{bgcolor=mintedBackgroundPHP} %startinline=True}
\SetupFloatingEnvironment{listing}{name=Code Sample}
\SetupFloatingEnvironment{listing}{listname=List of Code Samples}

\setlength{\parindent}{0pt} %
\setlength{\parskip}{.25cm}
\newcommand{\comment}[1]{}

\usepackage{amsmath}
\usepackage{algorithm2e}
\SetKwInOut{Input}{input}
\SetKwInOut{Output}{output}
%NOTE: you can embed algorithms in solutions, but they cannot be floating objects; use [H] to make them non-floats

\usepackage{lastpage}

%\usepackage{titling}
\usepackage{fancyhdr}
\renewcommand*{\titlepagestyle}{fancy}
\pagestyle{fancy}
%\fancyhf{}
%\rhead{Computer Science I}
%\lhead{Guides and tutorials}
\renewcommand{\headrulewidth}{0.0pt}
\renewcommand{\footrulewidth}{0.4pt}
\lfoot{\Title\ -- Computer Science I}
\cfoot{~}
\rfoot{\thepage\ / \pageref*{LastPage}}


\makeatletter
\title{Hack 9.0}\let\Title\@title
\subtitle{Debugging \\
Computer Science I\\
{\small
\vskip1cm
Department of Computer Science \& Engineering \\
University of Nebraska--Lincoln}
\vskip-1cm}
%\author{Dr.\ Chris Bourke}
\date{~}
\makeatother

\begin{document}

\maketitle

\hrule

\section*{Introduction}

Hack session activities are small weekly programming assignments intended
to get you started on full programming assignments.  Collaboration is allowed
and, in fact, \emph{highly encouraged}.  You may start on the activity before
your hack session, but during the hack session you must either be actively 
working on this activity or \emph{helping others} work on the activity.
You are graded using the same rubric as assignments so documentation, style, 
design and correctness are all important.  This activity is \textbf{due 
at 23:59:59 on the Friday} in the week in which it is assigned according 
to the CSE system clock.

\section*{Problem Statement}

Debugging is an essential skill for any software developer.  The novice
approach of using debug print statements is inefficient, error-prone, and
unreliable in general.  Real debugging requires a more sophisticated tool
called (surprisingly) a \emph{debugger}.  In this hack, we'll introduce you
to the basic usage of GDB, GNU's Debugger tool.

Starter code for this hack can be found in the following repository:

\url{https://github.com/cbourke/CSCE155-Hack9.0}

\section*{Basic Walkthrough}

We've provided a small program (see \mintinline{text}{arrayDebug.c}) 
that randomly generates an array of integers
and computes their average.  However, the program contains several bugs.
Follow the instructions below to use GDB to inspect this program and find
and fix these bugs.  To do this hack, you'll need to work on CSE or have 
GDB installed on your own computer.

\subsubsection*{Bug, The First}

\begin{enumerate}
  \item To get started, we'll need to compile the program specifically for GDB
  by using the \mintinline{text}{-g} flag:

  \mintinline{text}{gcc -g arrayDebug.c}

  Using this flag means that GDB will preserve function and variable names
  and line numbers for the debugger.  Normally this information is not 
  retained in a compiled program.  \emph{Note}: if you use the 
  \mintinline{text}{-Wall} flag, you may want to omit it in this walkthrough
  to avoid spoilers.
  
  \item Take a quick look at the code to get an idea of what the program
  is trying to compute, but do not change anything yet.  Try running the 
  program; it takes 1 command line argument: an integer
  representing the size of the random array you want to generate.  You'll
  want to keep it small, say: 
  
  \mintinline{text}{./a.out 5}

  You'll see that the program crashes with a segmentation fault.  There is
  little information to go on, so we'll need to fire up a debugger and see
  what's going on.
  
  \item To start the GNU debugger running your program with command line arguments
  you can use:
  
  \mintinline{text}{gdb --args a.out 5}
  
  Alternatively, you can start GDB without args using \mintinline{text}{gdb a.out}
  and once it has started, you can set (and reset) the argument(s) using 
  \mintinline{text}{set args 10}

  \item Once GDB has started, you'll be in an interactive ``shell'' in which
  you can give GDB (and your program) commands.  Start by having GDB run your
  program by typing: 
  
  \mintinline{text}{run}
  
  which will run your program and \emph{pause} its execution when it gets to the
  segmentation fault.  
  
  \item GDB will display the line of code at which the error occurred.
  However, it will be more helpful to see the entire source file.  To do
  this use the Text User Interface (TUI) mode by typing:
  
  \mintinline{text}{layout next}
  
  Using the up and down arrows, you can scroll through the source file.
  Occasionally, the display may glitch.  Don't panic, simply type:
  
  \mintinline{text}{refresh}
  
  and the screen will refresh itself.  As a short-cut you can use control-L
  to refresh.
  
  \item Examine some of the variable values to investigation what the
  issue is.  To print a variable value, use:
  
  \mintinline{text}{print i}

  \mintinline{text}{print n}
  
  To print the contents of an array, you need to dereference it and 
  provide GDB with its length.  Example:
  
  \mintinline{text}{print *a@5}
  
  In this particular case GDB will be unable to print the contents of the
  array.  Obviously something went wrong before the program got to this point.
  
  \item To find out what went wrong with the array, we need to rerun
  the program and examine its execution \emph{before} we got to the
  segmentation fault.  To do this, set a \emph{breakpoint} in the program.
  A breakpoint is a line or point in a program where the debugger will 
  pause execution from which you can continue execution in a step-by-step
  manner.  Set a breakpoint at the \mintinline{c}{randomArray} function 
  where the array was created:
  
  \mintinline{text}{break randomArray}
  
  Restart the program by typing \mintinline{text}{run} and confirming.
  
  \item GDB will pause execution at the beginning of the 
  \mintinline{c}{randomArray} function.  You can now step through
  the program line by line by typing \mintinline{text}{next} or 
  just simply \mintinline{text}{n}.  Note that if you want to resume
  execution, you can use the \mintinline{text}{continue} command.
  \begin{enumerate}
    \item Step through an iteration of the for loop and print the
    array: \mintinline{text}{print *a@5}
    \item Step through the second iteration and print it again:
    \mintinline{text}{print *a@5}
    \item This can get tedious, so you can set a watchpoint that
    will print a variable any time it changes.  Set a watch point for
    the array: \mintinline{text}{watch *a@5}
    \item Step through the rest of the iterations and watch the
    array get filled with random values. 
    \item Step through the return of the function; when it gets back
    to the \mintinline{c}{main} function, print the contents
    of the array again: \mintinline{text}{print *arr@5} and you'll 
    see that you cannot access the memory there.  
    %TODO: consider adding the finish command here
  \end{enumerate}

  \item Quit out of GDB using \mintinline{text}{quit} and edit
  the code to fix this bug (hint: what did the \mintinline{c}{randomArray}
  return?) and recompile.
  
\end{enumerate}

\subsubsection*{Bug, The Sequel}

\begin{enumerate}
  \item Run the program again from the command line to see if it works.
  It should run but it probably won't give correct results (on some runs it
  may, but on most it will not).  Start GDB again
  but this time use the following to immediately start it in TUI
  
  \mintinline{text}{gdb -tui --args a.out 5}
  
  \item The problem probably lies with how we're computing the average, so 
  set a breakpoint either at the start of the \mintinline{c}{average} function:
  
  \mintinline{text}{break average}
  
  or at a particular line within that function, example:
  
  \mintinline{text}{break 54}
  
  \item Set another break point at the end of the function:

  \mintinline{text}{break 58}
  
  \item Now, you can either step through the execution of the loop 
  and/or set a watchpoint on the \mintinline{c}{sum} variable or you 
  can skip ahead to the end of the loop by using \mintinline{text}{continue}
  which will resume execution until the next breakpoint.
  
  \item Print the \mintinline{c}{sum} variable's value to see if it 
  matches the value you expect.  Or:
  
  \begin{itemize}
    \item You can print all local variables using
    
    \mintinline{c}{info locals}
  
     \item You can print all the parameter values using
     
     \mintinline{c}{info args}

	 \item You can print the value of expressions as well, example:
	 
	 \mintinline{c}{print sum / n}
	 
	 \item Since return values are not stored in variables, another thing
	 you can do is to use 

	 \mintinline{c}{finish}
	 
	 which will continue execution until the function returns and then
	 prints the returned value.
  \end{itemize}

  \item Fix the bug(s) that you were able to identify and recompile/rerun the
  program until you are certain that all bugs have been addressed.
  
\end{enumerate}


\subsection*{The Main Show}

Now that you have some familiarity with GDB, you'll use your new debugging
skills to debug a full program.  We have provided source files and a 
makefile that builds a tic-tac-toe game project.  The game allows a player
to select several different modes including a 2-player game, a 1-player 
versus the computer (which makes random moves) and a 1-player versus 
a ``smart'' computer.

Similar to the previous part, we have introduced several bugs (6 to be exact).
Run the program and use GDB to walk through section(s) of the code you believe
to contain bugs and fix them all until the program runs without any problems.

Some tips/hints:
\begin{itemize}
  \item Write down the sequence of inputs/moves that led to a crash/bug 
  so that you can \emph{reproduce} the sequence when debugging and so that
  you can retest the sequence when you think you've fixed the bug.
  \item When you observe a bug, describe exactly how it manifests, this 
  will help you form a \emph{hypothesis} 
  \item Set breakpoints and watchpoints so that you can trace through
  the program to points where you believe bugs might be located.
  \item Try to fix only one problem at a time and keep track of your
  fixes.  In a real project you would do so through a bug tracking system
  and commit messages, but for this hack you may do so through comments.
  \item Keep in mind that bugs may manifest themselves later in the program
  when the problem lies earlier in the program.
  \item You can get a bigger picture in GDB by printing out a full stack trace
  using 
  
  \mintinline{text}{backtrace full}

\end{itemize}  

\section*{Instructions}

\begin{itemize}
  \item You are encouraged to collaborate any number of students 
  before, during, and after your scheduled hack session.  
  \item Handin all the (corrected) files to webhandin and use
  the webgrader to verify that all bugs have been fixed.
  \item Each individual student will need
  to hand in their own copy and will receive their own individual
  grade.
\end{itemize}

% Pay no attention to the man behind the curtain
%\section*{Bug List}
%
%\begin{enumerate}
%  
%  \item gameUtils.c line 55 should be \mintinline{c}{i<2} otherwise an extraneous line is printed
%  \item gameUtils.c line 115 should be \mintinline{c}{board[i][j] == NONE}:
%    Otherwise, it will immediately end the game because it is not a "playing" status.
%  \item ticTacToeDriver.c line 29 should be \mintinline{c}{userMove(board, O);} (Oh, not zero): causes 2nd player's input is ignored and gets stuck in a loop unless that player selects an already chosen option, at which it goes back to player 1.
%  \item playerUtils.c line 19: should be \mintinline{c}{board[(input-1)/3][(input-1)%3] = player;} otherwise only choices 1-3 work, choices 4-9 are ignored
%  \item gameUtils.c line 85: should be \mintinline{c}{(board[0][0] == board[0][1] && board[0][1] == board[0][2] && board[0][0] == X)}; the indices being off only checks the upper left for a win condition instead of the entire row.
%  \item gameUtils.c line 23 should be \mintinline{c}{return copy;}; causes segfaults when free is called.
%
%\end{enumerate}


\end{document}
