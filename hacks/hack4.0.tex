\documentclass[12pt]{scrartcl}


\usepackage{epsfig,amssymb}

\usepackage{xcolor}
\usepackage{graphicx}
\usepackage{epstopdf}
\usepackage{multirow}

\definecolor{darkred}{rgb}{0.5,0,0}
\definecolor{darkgreen}{rgb}{0,0.5,0}
\usepackage[pdfusetitle]{hyperref}
\hypersetup{
  letterpaper,
  colorlinks,
  linkcolor=red,
  citecolor=darkgreen,
  menucolor=darkred,
  urlcolor=blue,
  pdfpagemode=none,
}

\usepackage{fullpage}
\usepackage{tikz}
\pagestyle{empty} %
\usepackage{subfigure}

\definecolor{MyDarkBlue}{rgb}{0,0.08,0.45}
\definecolor{MyDarkRed}{rgb}{0.45,0.08,0}
\definecolor{MyDarkGreen}{rgb}{0.08,0.45,0.08}

\definecolor{mintedBackground}{rgb}{0.95,0.95,0.95}
\definecolor{mintedInlineBackground}{rgb}{.90,.90,1}

\usepackage[newfloat=true]{minted}

\setminted{mathescape,
           linenos,
           autogobble,
           frame=none,
           framesep=2mm,
           framerule=0.4pt,
           %label=foo,
           xleftmargin=2em,
           xrightmargin=0em,
           %startinline=true,  %PHP only, allow it to omit the PHP Tags *** with this option, variables using dollar sign in comments are treated as latex math
           numbersep=10pt, %gap between line numbers and start of line
           style=default} %syntax highlighting style, default is "default"

\setmintedinline{bgcolor={mintedBackground}}
%doesn't work with the above workaround:
\setminted{bgcolor={mintedBackground}}
\setminted[text]{bgcolor={mintedBackground},linenos=false,autogobble,xleftmargin=1em}
%\setminted[php]{bgcolor=mintedBackgroundPHP} %startinline=True}
\SetupFloatingEnvironment{listing}{name=Code Sample}
\SetupFloatingEnvironment{listing}{listname=List of Code Samples}

\setlength{\parindent}{0pt} %
\setlength{\parskip}{.25cm}
\newcommand{\comment}[1]{}

\usepackage{amsmath}
\usepackage{algorithm2e}
\SetKwInOut{Input}{input}
\SetKwInOut{Output}{output}
%NOTE: you can embed algorithms in solutions, but they cannot be floating objects; use [H] to make them non-floats

\usepackage{lastpage}

%\usepackage{titling}
\usepackage{fancyhdr}
\renewcommand*{\titlepagestyle}{fancy}
\pagestyle{fancy}
%\fancyhf{}
%\rhead{Computer Science I}
%\lhead{Guides and tutorials}
\renewcommand{\headrulewidth}{0.0pt}
\renewcommand{\footrulewidth}{0.4pt}
\lfoot{\Title\ -- Computer Science I}
\cfoot{~}
\rfoot{\thepage\ / \pageref*{LastPage}}


\makeatletter
\title{Hack 4.0}\let\Title\@title
\subtitle{Computer Science I\\
{\small
\vskip1cm
Department of Computer Science \& Engineering \\
University of Nebraska--Lincoln}
\vskip-1cm}
%\author{Dr.\ Chris Bourke}
\date{~}
\makeatother

\begin{document}

\maketitle

\hrule

\section*{Introduction}

Hack session activities are small weekly programming assignments intended
to get you started on full programming assignments.  Collaboration is allowed
and, in fact, \emph{highly encouraged}.  You may start on the activity before
your hack session, but during the hack session you must either be actively 
working on this activity or \emph{helping others} work on the activity.
You are graded using the same rubric as assignments so documentation, style, 
design and correctness are all important.

%\subsection*{Rubric}
%\begin{table}[H]
%\begin{tabular}{ll}
%Category       & Point Value \\
%Style          & 2           \\
%Documentation  & 2           \\
%Design         & 5           \\
%Correctness    & 16          \\
%\textbf{Total} & \textbf{25}
%\end{tabular}
%\end{table}



\section*{Problem Statement}

A 401(k) plan is a type of tax-qualified account for people to save 
for retirement.  Employees typically have part of their pre-tax pay
check deposited into such an account and employers may match a certain
amount of those contributions.  As of 2018, there is a maximum
annual contribution limit of \$18,500.
 
The money contributed to a 401(k) account may be invested in many
different ways.  While the account may grow tax free, it is not 
immune to inflation.  To account for inflation, you can use
the following formula which is the inflation-adjusted rate of
return.
  $$\frac{1 + \textrm{rate of return}}{1+\textrm{inflation rate}} - 1$$

Write a program that produces an \emph{amortization table} for a
401(k) account.  Your program will read the following inputs as 
command line arguments.
\begin{itemize}
  \item An initial starting balance
  \item A monthly contribution amount (we'll assume its the same over the life of the savings plan)
  \item An (average) annual rate of return (on the scale $[0, 1]$)
  \item An (average) annual rate of inflation (on the scale $[0, 1]$)
  \item A number of years until retirement
\end{itemize}
Your program will then compute a monthly savings table detailing
the (inflation-adjusted) interest earned each month, contribution, and
new balance.  To get the monthly rate, simply divide by 12.  Each month, 
interest is applied to the balance at this rate (prior to the monthly 
deposit) and the monthly contribution is added.  Thus, the earnings 
compound month to month.  Be sure to round appropriately, to the nearest
cent so that rounding errors do not compound.

Your program will need to validate the input and quit with an error
message on any potential bad input(s).  For inputs 

\mintinline{text}{10000 500 0.09 0.012 10} 

your output should look something like the following:

\begin{figure}[H]
\begin{minted}{text}
Month    Interest     Balance 
    1 $     64.23 $  10564.23
    2 $     67.85 $  11132.08
    3 $     71.50 $  11703.58
    4 $     75.17 $  12278.75
    5 $     78.87 $  12857.62
    6 $     82.58 $  13440.20
    7 $     86.33 $  14026.53
    8 $     90.09 $  14616.62
    9 $     93.88 $  15210.50
...    
  116 $    678.19 $ 106767.24
  117 $    685.76 $ 107953.00
  118 $    693.37 $ 109146.37
  119 $    701.04 $ 110347.41
  120 $    708.75 $ 111556.16
Total Interest Earned: $  41556.16
Total Nest Egg: $ 111556.16
\end{minted}
\end{figure}

\section*{Instructions}

\begin{itemize}
  \item You are encouraged to collaborate any number of students 
  before, during, and after your scheduled hack session.  
  \item Design at least 3 test cases \emph{before} you begin
  designing or implementing your program.  Test cases are 
  input-output pairs that are known to be correct using means
  other than your program.
  \item Include the name(s) of everyone who worked together on
  this activity in your source file's header.
  \item Name your program \mintinline{text}{retire.c}, and
  turn it in via webhandin, making sure that it runs and executes
  correctly in the webgrader.  Each individual student will need
  to hand in their own copy and will receive their own individual
  grade.
\end{itemize}
  


\end{document}
