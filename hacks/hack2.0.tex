\documentclass[12pt]{scrartcl}


\usepackage{epsfig,amssymb}

\usepackage{xcolor}
\usepackage{graphicx}
\usepackage{epstopdf}
\usepackage{multirow}

\definecolor{darkred}{rgb}{0.5,0,0}
\definecolor{darkgreen}{rgb}{0,0.5,0}
\usepackage[pdfusetitle]{hyperref}
\hypersetup{
  letterpaper,
  colorlinks,
  linkcolor=red,
  citecolor=darkgreen,
  menucolor=darkred,
  urlcolor=blue,
  pdfpagemode=none,
}

\usepackage{fullpage}
\usepackage{tikz}
\pagestyle{empty} %
\usepackage{subfigure}

\definecolor{MyDarkBlue}{rgb}{0,0.08,0.45}
\definecolor{MyDarkRed}{rgb}{0.45,0.08,0}
\definecolor{MyDarkGreen}{rgb}{0.08,0.45,0.08}

\definecolor{mintedBackground}{rgb}{0.95,0.95,0.95}
\definecolor{mintedInlineBackground}{rgb}{.90,.90,1}

\usepackage[newfloat=true]{minted}

\setminted{mathescape,
           linenos,
           autogobble,
           frame=none,
           framesep=2mm,
           framerule=0.4pt,
           %label=foo,
           xleftmargin=2em,
           xrightmargin=0em,
           %startinline=true,  %PHP only, allow it to omit the PHP Tags *** with this option, variables using dollar sign in comments are treated as latex math
           numbersep=10pt, %gap between line numbers and start of line
           style=default} %syntax highlighting style, default is "default"

\setmintedinline{bgcolor={mintedBackground}}
%doesn't work with the above workaround:
\setminted{bgcolor={mintedBackground}}
\setminted[text]{bgcolor={mintedBackground},linenos=false,autogobble,xleftmargin=1em}
%\setminted[php]{bgcolor=mintedBackgroundPHP} %startinline=True}
\SetupFloatingEnvironment{listing}{name=Code Sample}
\SetupFloatingEnvironment{listing}{listname=List of Code Samples}

\setlength{\parindent}{0pt} %
\setlength{\parskip}{.25cm}
\newcommand{\comment}[1]{}

\usepackage{amsmath}
\usepackage{algorithm2e}
\SetKwInOut{Input}{input}
\SetKwInOut{Output}{output}
%NOTE: you can embed algorithms in solutions, but they cannot be floating objects; use [H] to make them non-floats

\usepackage{lastpage}

%\usepackage{titling}
\usepackage{fancyhdr}
\renewcommand*{\titlepagestyle}{fancy}
\pagestyle{fancy}
%\fancyhf{}
%\rhead{Computer Science I}
%\lhead{Guides and tutorials}
\renewcommand{\headrulewidth}{0.0pt}
\renewcommand{\footrulewidth}{0.4pt}
\lfoot{\Title\ -- Computer Science I}
\cfoot{~}
\rfoot{\thepage\ / \pageref*{LastPage}}


\makeatletter
\title{Hack 2.0}\let\Title\@title
\subtitle{Computer Science I\\
{\small
\vskip1cm
Department of Computer Science \& Engineering \\
University of Nebraska--Lincoln}
\vskip-1cm}
%\author{Dr.\ Chris Bourke}
\date{~}
\makeatother

\begin{document}

\maketitle

\hrule

\section*{Introduction}

Hack session activities are small weekly programming assignments intended
to get you started on full programming assignments.  Collaboration is allowed
and, in fact, \emph{highly encouraged}.  You may start on the activity before
your hack session, but during the hack session you must either be actively 
working on this activity or \emph{helping others} work on the activity.
You are graded using the same rubric as assignments so documentation, style, 
design and correctness are all important.  This activity is \textbf{due 
at 23:59:59 on the Friday} in the week in which it is assigned according 
to the CSE system clock.

%TODO: rubric 
% instructions
% documentation
% style
% design including input validation
% correctness

\section*{Problem Statement}

Consider two locations, an origin and a destination, on the globe 
identified by their latitude and longitude.  The distance between
these two locations can be computed using the Spherical Law of 
Cosines.  In particular, the distance $d$ is
 $$d = \arccos{(\sin(\varphi_1) \sin(\varphi_2) + \cos(\varphi_1) \cos(\varphi_2) \cos(\Delta) )} \cdot R$$
where
\begin{itemize}
  \item $\varphi_1$ is the latitude of location $A$, $\varphi_2$ is the latitude of location $B$
  \item $\Delta$ is the difference between location $B$'s longitude and location $A$'s longitude
  \item $R$ is the (average) radius of the earth, 6,371 kilometers
\end{itemize}

Write a program that reads in the latitude and longitude of two locations
as command line arguments and then computes the distance between them using
the above formula.  Note that latitude inputs will be in degrees and in the
range $[-90, 90]$ and longitude will be in degrees in the range $[-180, 180]$.
Negative values correspond to the western and southern hemispheres.  

Note that the formula above assumes that latitude and longitude are 
measured in radians $r$, $-\pi \leq r \leq \pi$.  You can convert from
degrees $deg$ to radians $r$ using the formula
  $$r = \frac{deg}{180} \cdot \pi$$  

Your program output should look something like the following.  

\begin{minted}{text}
Location Distance
========================
Origin:      (41.948300, -87.655600)
Destination: (40.820600, -96.705600)
Air distance is 764.990931 kms
\end{minted}


\section*{Instructions}

\begin{itemize}
  \item You are encouraged to collaborate any number of students 
  before, during, and after your scheduled hack session.  
  \item Design at least 3 test cases \emph{before} you begin
  designing or implementing your program.  Test cases are 
  input-output pairs that are known to be correct using means
  other than your program.
  \item Include the name(s) of everyone who worked together on
  this activity in your source file's header.
  \item Name your program \mintinline{text}{airDistance.c}, and
  turn it in via webhandin, making sure that it runs and executes
  correctly in the webgrader.  Each individual student will need
  to hand in their own copy and will receive their own individual
  grade.
\end{itemize}
  


\end{document}
