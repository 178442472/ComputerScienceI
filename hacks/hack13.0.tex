\documentclass[12pt]{scrartcl}


\usepackage{epsfig,amssymb}

\usepackage{xcolor}
\usepackage{graphicx}
\usepackage{epstopdf}
\usepackage{multirow}

\definecolor{darkred}{rgb}{0.5,0,0}
\definecolor{darkgreen}{rgb}{0,0.5,0}
\usepackage[pdfusetitle]{hyperref}
\hypersetup{
  letterpaper,
  colorlinks,
  linkcolor=red,
  citecolor=darkgreen,
  menucolor=darkred,
  urlcolor=blue,
  pdfpagemode=none,
}

\usepackage{fullpage}
\usepackage{tikz}
\pagestyle{empty} %
\usepackage{subfigure}

\definecolor{MyDarkBlue}{rgb}{0,0.08,0.45}
\definecolor{MyDarkRed}{rgb}{0.45,0.08,0}
\definecolor{MyDarkGreen}{rgb}{0.08,0.45,0.08}

\definecolor{mintedBackground}{rgb}{0.95,0.95,0.95}
\definecolor{mintedInlineBackground}{rgb}{.90,.90,1}

\usepackage[newfloat=true]{minted}

\setminted{mathescape,
           linenos,
           autogobble,
           frame=none,
           framesep=2mm,
           framerule=0.4pt,
           %label=foo,
           xleftmargin=2em,
           xrightmargin=0em,
           %startinline=true,  %PHP only, allow it to omit the PHP Tags *** with this option, variables using dollar sign in comments are treated as latex math
           numbersep=10pt, %gap between line numbers and start of line
           style=default} %syntax highlighting style, default is "default"

\setmintedinline{bgcolor={mintedBackground}}
%doesn't work with the above workaround:
\setminted{bgcolor={mintedBackground}}
\setminted[text]{bgcolor={mintedBackground},linenos=false,autogobble,xleftmargin=1em}
%\setminted[php]{bgcolor=mintedBackgroundPHP} %startinline=True}
\SetupFloatingEnvironment{listing}{name=Code Sample}
\SetupFloatingEnvironment{listing}{listname=List of Code Samples}

\setlength{\parindent}{0pt} %
\setlength{\parskip}{.25cm}
\newcommand{\comment}[1]{}

\usepackage{amsmath}
\usepackage{algorithm2e}
\SetKwInOut{Input}{input}
\SetKwInOut{Output}{output}
%NOTE: you can embed algorithms in solutions, but they cannot be floating objects; use [H] to make them non-floats

\usepackage{lastpage}

%\usepackage{titling}
\usepackage{fancyhdr}
\renewcommand*{\titlepagestyle}{fancy}
\pagestyle{fancy}
%\fancyhf{}
%\rhead{Computer Science I}
%\lhead{Guides and tutorials}
\renewcommand{\headrulewidth}{0.0pt}
\renewcommand{\footrulewidth}{0.4pt}
\lfoot{\Title\ -- Computer Science I}
\cfoot{~}
\rfoot{\thepage\ / \pageref*{LastPage}}


\makeatletter
\title{Hack 13.0}\let\Title\@title
\subtitle{Computer Science I\\
Searching \& Sorting\\
{\small
\vskip1cm
Department of Computer Science \& Engineering \\
University of Nebraska--Lincoln}
\vskip-3cm}
%\author{Dr.\ Chris Bourke}
\date{~}
\makeatother

\begin{document}

\maketitle

\hrule

\section*{Introduction}

Hack session activities are small weekly programming assignments intended
to get you started on full programming assignments.  Collaboration is allowed
and, in fact, \emph{highly encouraged}.  You may start on the activity before
your hack session, but during the hack session you must either be actively 
working on this activity or \emph{helping others} work on the activity.
You are graded using the same rubric as assignments so documentation, style, 
design and correctness are all important.  This activity is \textbf{due 
at 23:59:59 on the Friday} in the week in which it is assigned according 
to the CSE system clock.

\section*{Problem Statement}

You will use the implementation of your Airport data model from a previous
hack to develop several reports that will require you to sort and search
(a subset of) the International Civil Aviation Organization database for
particular airports.  

We have provided an updated header file, \mintinline{text}{airport.h} with
the specific functions that you need to implement.  First, you'll need to
implement 8 different comparator functions for your \mintinline{c}{Airport}
structure.  The details of the expected orders are documented in the header
file.  

Second, you'll need to write code that produces several reports.  All your
code needs to be placed into the following function:

\mintinline{c}{void generateReports(Airport *airports, int n);}

which takes an array of \mintinline{c}{Airport} structures and produces the
following reports which should all be output to the standard output.

\begin{itemize}
  \item To help you troubleshoot, you should print out all the structures
  in the original order.
  \item Sort the airports by each of the 8 comparators and print them 
  out (8 reports total)
  \item Search for and print out the airport in the array that is closest (via 
  air distance) to Lincoln.  Lincoln is located at 40.8507N, 96.7581W.
  \item Search for and print out the airport that is the geographic west-east
  median of the given airports with respect to longitude.  
  \item Search for an airport located in the city New York and print it out
  if it exists.  If no such airport exists, print out an appropriate message.
  \item Search for an airport located in Canada (country code ``CA'') and print it out
  if it exists.  If no such airport exists, print out an appropriate message.
  \item Search for an airport whose type is \mintinline{text}{large_airport} and
  print it out if it exists.  If no such airport exists, print out an appropriate message.
\end{itemize}

\section*{Instructions}

\begin{itemize}

  \item Place all of your function definitions in a source file named 
  \mintinline{text}{airport.c} and hand it in with your header file, 
  \mintinline{text}{airport.h}.  You may add any utility functions you
  wish but you must \emph{not} change any of the signatures of the required
  functions.
  
  \item In addition, you must create a main test driver program that 
  demonstrates your reports with at least 5 airports. Name this file 
  \mintinline{text}{airportReport.c} and hand it in.

  \item You are encouraged to collaborate any number of students 
  before, during, and after your scheduled hack session.  

  \item You may (in fact are encouraged) to define any additional
  ``helper'' functions that may help you.

  \item Include the name(s) of everyone who worked together on
  this activity in your source file's header.

  \item Turn in all of your files via webhandin, making sure that 
  it runs and executes correctly in the webgrader.  Each individual 
  student will need to hand in their own copy and will receive 
  their own individual grade.
\end{itemize}  


\end{document}
