\documentclass[12pt]{scrartcl}


\usepackage{epsfig,amssymb}

\usepackage{xcolor}
\usepackage{graphicx}
\usepackage{epstopdf}
\usepackage{multirow}

\definecolor{darkred}{rgb}{0.5,0,0}
\definecolor{darkgreen}{rgb}{0,0.5,0}
\usepackage[pdfusetitle]{hyperref}
\hypersetup{
  letterpaper,
  colorlinks,
  linkcolor=red,
  citecolor=darkgreen,
  menucolor=darkred,
  urlcolor=blue,
  pdfpagemode=none,
}

\usepackage{fullpage}
\usepackage{tikz}
\pagestyle{empty} %
\usepackage{subfigure}

\definecolor{MyDarkBlue}{rgb}{0,0.08,0.45}
\definecolor{MyDarkRed}{rgb}{0.45,0.08,0}
\definecolor{MyDarkGreen}{rgb}{0.08,0.45,0.08}

\definecolor{mintedBackground}{rgb}{0.95,0.95,0.95}
\definecolor{mintedInlineBackground}{rgb}{.90,.90,1}

\usepackage[newfloat=true]{minted}

\setminted{mathescape,
           linenos,
           autogobble,
           frame=none,
           framesep=2mm,
           framerule=0.4pt,
           %label=foo,
           xleftmargin=2em,
           xrightmargin=0em,
           %startinline=true,  %PHP only, allow it to omit the PHP Tags *** with this option, variables using dollar sign in comments are treated as latex math
           numbersep=10pt, %gap between line numbers and start of line
           style=default} %syntax highlighting style, default is "default"

\setmintedinline{bgcolor={mintedBackground}}
%doesn't work with the above workaround:
\setminted{bgcolor={mintedBackground}}
\setminted[text]{bgcolor={mintedBackground},linenos=false,autogobble,xleftmargin=1em}
%\setminted[php]{bgcolor=mintedBackgroundPHP} %startinline=True}
\SetupFloatingEnvironment{listing}{name=Code Sample}
\SetupFloatingEnvironment{listing}{listname=List of Code Samples}

\setlength{\parindent}{0pt} %
\setlength{\parskip}{.25cm}
\newcommand{\comment}[1]{}

\usepackage{amsmath}
\usepackage{algorithm2e}
\SetKwInOut{Input}{input}
\SetKwInOut{Output}{output}
%NOTE: you can embed algorithms in solutions, but they cannot be floating objects; use [H] to make them non-floats

\usepackage{lastpage}

%\usepackage{titling}
\usepackage{fancyhdr}
\renewcommand*{\titlepagestyle}{fancy}
\pagestyle{fancy}
%\fancyhf{}
%\rhead{Computer Science I}
%\lhead{Guides and tutorials}
\renewcommand{\headrulewidth}{0.0pt}
\renewcommand{\footrulewidth}{0.4pt}
\lfoot{\Title\ -- Computer Science I}
\cfoot{~}
\rfoot{\thepage\ / \pageref*{LastPage}}


\makeatletter
\title{Hack 6.0}\let\Title\@title
\subtitle{Computer Science I\\
{\small
\vskip1cm
Department of Computer Science \& Engineering \\
University of Nebraska--Lincoln}
\vskip-1cm}
%\author{Dr.\ Chris Bourke}
\date{~}
\makeatother

\begin{document}

\maketitle

\hrule

\section*{Introduction}

Hack session activities are small weekly programming assignments intended
to get you started on full programming assignments.  Collaboration is allowed
and, in fact, \emph{highly encouraged}.  You may start on the activity before
your hack session, but during the hack session you must either be actively 
working on this activity or \emph{helping others} work on the activity.
You are graded using the same rubric as assignments so documentation, style, 
design and correctness are all important.  This activity is \textbf{due 
at 23:59:59 on the Friday} in the week in which it is assigned according 
to the CSE system clock.

%TODO: rubric 
% instructions
% documentation
% style
% design including input validation
% correctness

\section*{Problem Statement}

In this hack you'll get some more practice writing functions that utilize 
pass-by-reference (pointers), error handling and enumerated types.  There 
are several different
ways to model colors including RGB and CMYK.  RGB is generally used in displays
and models a color with three values in the range $[0, 255]$ corresponding to 
the red, green and blue ``contribution'' to the color.  For example, the
triple $(255, 255, 0)$ corresponds to a full red and green (additive) value
which results in yellow.  CMYK or Cyan-Magenta-Yellow-Black is a model used
in printing where four colors of ink are combined to make various colors.
In this system, the four values are on the scale $[0, 1]$.  Write several
functions to convert between these models as well as converting an RGB
color to a \emph{grayscale} value.

\begin{enumerate}
\item Write a function to convert from an RGB color model to CMYK.  To 
convert to CMYK, you first need to scale each integer value to the range 
$[0, 1]$ by simply computing
	$$r' = \frac{r}{255}, \quad g' = \frac{g}{255}, \quad b' = \frac{b}{255}$$
	and then using the following formulas:
\begin{align*}
k & = 1-\max\{r', g', b'\} \\
c & = \frac{(1-r'-k)}{(1-k)} \\
m & = \frac{(1-g'-k)}{(1-k)} \\
y & = \frac{(1-b'-k)}{(1-k)} \\
\end{align*}
Your function should have the following signature:

\mintinline{c}{int rgbToCMYK(int r, int g, int b, double *c, double *m, double *y double *k)}

Identify any and all error conditions and use the return value to indicate
an error code (0 for no error, non-zero value(s) for error conditions).

\item Write a function to convert from CMYK to RGB using the following formulas.
\begin{align*}
r & = 255 \cdot (1 - c) \cdot (1-k) \\
g & = 255 \cdot (1 - m) \cdot (1-k) \\
b & = 255 \cdot (1 - y) \cdot (1-k) \\
\end{align*}
Your function should have the following signature:

\mintinline{c}{int cmykToRGB(double c, double m, double y double k, int *r, int *g, int *b)}

Identify any and all error conditions and use the return value to indicate
an error code (0 for no error, non-zero value(s) for error conditions).

\item Write a function that takes an RGB color value and transforms it 
to a grayscale version, ``removing'' the color values.  An RGB color 
value is grayscale if all three components have the same value.  To 
transform a color value to grayscale, there there are several possible 
techniques.  The average method simply sets all three values to the 
average:
  $$\frac{r + g + b}{3}$$
The lightness method averages the most prominent and least prominent 
colors:
  $$\frac{\max\{r, g, b\} + \min\{r, g, b\}}{2}$$
The luminosity technique uses a weighted average to account for a human 
perceptual preference toward green, setting all three values to:
  $$0.21 r + 0.72 g + 0.07 b$$
In all three cases, the integral values should be \emph{rounded} rather 
than truncated.

Your function should have the following signature:

\mintinline{c}{int toGrayScale(int *r, int *g, int *b, Mode m)}

To specify which of the three \emph{modes} is to be used, define
and use the following enumerated type in your header file:
\begin{minted}{c}
typedef enum {
  AVERAGE,
  LIGHTNESS,
  LUMINOSITY
} Mode;
\end{minted}
Identify any and all error conditions and use the return value to indicate
an error code (0 for no error, non-zero value(s) for error conditions).

\end{enumerate}



\section*{Instructions}
	
Place all of your function prototypes into a file named \mintinline{c}{utils.h} and and their definitions in a
file named \mintinline{c}{utils.c}.  In addition, create a main test driver program that demonstrates at
least 3 test calls to each of your functions to verify that their output matches your expected
output.  You should \emph{not} prompt for input; rather your test cases should be hardcoded in
your test driver's main function.  Place your main function into a file named \mintinline{c}{utilsTester.c}.  
Hand in all three of your source files using webhandin.


\begin{itemize}
  \item You are encouraged to collaborate any number of students 
  before, during, and after your scheduled hack session.  
  \item Design at least 3 test cases for each function
  \emph{before} you begin
  designing or implementing your program.  Test cases are 
  input-output pairs that are known to be correct using means
  other than your program.
  \item You may (in fact are encouraged) to define any additional
  ``helper'' functions that may help you.
  \item Include the name(s) of everyone who worked together on
  this activity in your source file's header.
  \item Place your prototypes and documentation in a header file 
  named \mintinline{text}{colorUtils.h} and your source in a file
  named \mintinline{text}{colorUtils.c}.
  \item In addition, implement all of your test cases using 
  cmocka (\url{https://cmocka.org/}), a unit testing framework 
  for C.  An start file, 
  \mintinline{text}{utilsTester.c} has been provided which
  gives several examples of tests for the \mintinline{c}{rgbToCMYK}
  function.  You should add your own as well as use the 
  rest of the code as a guide for implementing tests for the
  other two functions.  The starter file should be enough, but
  the full documentation can be found here: \url{https://api.cmocka.org/}.
  A \mintinline{text}{makefile} has also been provided to help
  you easily compile your files.  
  \item Turn in all of your files via webhandin, making sure that 
  it runs and executes correctly in the webgrader.  Each individual 
  student will need to hand in their own copy and will receive 
  their own individual grade.
\end{itemize}  


\end{document}
