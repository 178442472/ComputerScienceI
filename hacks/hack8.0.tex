\documentclass[12pt]{scrartcl}


\usepackage{epsfig,amssymb}

\usepackage{xcolor}
\usepackage{graphicx}
\usepackage{epstopdf}
\usepackage{multirow}

\definecolor{darkred}{rgb}{0.5,0,0}
\definecolor{darkgreen}{rgb}{0,0.5,0}
\usepackage[pdfusetitle]{hyperref}
\hypersetup{
  letterpaper,
  colorlinks,
  linkcolor=red,
  citecolor=darkgreen,
  menucolor=darkred,
  urlcolor=blue,
  pdfpagemode=none,
}

\usepackage{fullpage}
\usepackage{tikz}
\pagestyle{empty} %
\usepackage{subfigure}

\definecolor{MyDarkBlue}{rgb}{0,0.08,0.45}
\definecolor{MyDarkRed}{rgb}{0.45,0.08,0}
\definecolor{MyDarkGreen}{rgb}{0.08,0.45,0.08}

\definecolor{mintedBackground}{rgb}{0.95,0.95,0.95}
\definecolor{mintedInlineBackground}{rgb}{.90,.90,1}

\usepackage[newfloat=true]{minted}

\setminted{mathescape,
           linenos,
           autogobble,
           frame=none,
           framesep=2mm,
           framerule=0.4pt,
           %label=foo,
           xleftmargin=2em,
           xrightmargin=0em,
           %startinline=true,  %PHP only, allow it to omit the PHP Tags *** with this option, variables using dollar sign in comments are treated as latex math
           numbersep=10pt, %gap between line numbers and start of line
           style=default} %syntax highlighting style, default is "default"

\setmintedinline{bgcolor={mintedBackground}}
%doesn't work with the above workaround:
\setminted{bgcolor={mintedBackground}}
\setminted[text]{bgcolor={mintedBackground},linenos=false,autogobble,xleftmargin=1em}
%\setminted[php]{bgcolor=mintedBackgroundPHP} %startinline=True}
\SetupFloatingEnvironment{listing}{name=Code Sample}
\SetupFloatingEnvironment{listing}{listname=List of Code Samples}

\setlength{\parindent}{0pt} %
\setlength{\parskip}{.25cm}
\newcommand{\comment}[1]{}

\usepackage{amsmath}
\usepackage{algorithm2e}
\SetKwInOut{Input}{input}
\SetKwInOut{Output}{output}
%NOTE: you can embed algorithms in solutions, but they cannot be floating objects; use [H] to make them non-floats

\usepackage{lastpage}

%\usepackage{titling}
\usepackage{fancyhdr}
\renewcommand*{\titlepagestyle}{fancy}
\pagestyle{fancy}
%\fancyhf{}
%\rhead{Computer Science I}
%\lhead{Guides and tutorials}
\renewcommand{\headrulewidth}{0.0pt}
\renewcommand{\footrulewidth}{0.4pt}
\lfoot{\Title\ -- Computer Science I}
\cfoot{~}
\rfoot{\thepage\ / \pageref*{LastPage}}


\makeatletter
\title{Hack 8.0}\let\Title\@title
\subtitle{Strings \& String Processing\\
Computer Science I\\
{\small
\vskip1cm
Department of Computer Science \& Engineering \\
University of Nebraska--Lincoln}
\vskip-1cm}
%\author{Dr.\ Chris Bourke}
\date{~}
\makeatother

\begin{document}

\maketitle

\hrule

\section*{Introduction}

Hack session activities are small weekly programming assignments intended
to get you started on full programming assignments.  Collaboration is allowed
and, in fact, \emph{highly encouraged}.  You may start on the activity before
your hack session, but during the hack session you must either be actively 
working on this activity or \emph{helping others} work on the activity.
You are graded using the same rubric as assignments so documentation, style, 
design and correctness are all important.  This activity is \textbf{due 
at 23:59:59 on the Friday} in the week in which it is assigned according 
to the CSE system clock.

\section*{Problem Statement}

\subsection*{Exercises}

To get more practice working with strings, you will write several 
functions that involve operations on strings.  In particular, implement
the following functions with the described behavior.  You \emph{must}
use the given signatures.

\begin{enumerate}

  \item Write a function that replaces instances of a given character 
  with a different character in a string.\\
  \mintinline{c}{void replaceChar(char *s, char oldChar, char newChar);}\\
  Which will replace any instance of the character stored in 
  \mintinline{c}{oldChar} with the character stored in \mintinline{c}{newChar} 
  in the string \mintinline{c}{s}.  

  \item Write a function that takes a string and creates a new copy of it
  but with instances of a given character replaced with a different character.\\
  \mintinline{c}{char * replaceCharCopy(const char *s, char oldChar, char newChar);}\\

  \item Write a function that takes a string and removes all instances 
  of a certain character from it.\\
  \mintinline{c}{void removeChar(char *s, char c);}\\
  When removing characters, all subsequent characters should be 
  shifted down.  Take care that you handle the null terminating character 
  properly.

  \item Write a function that takes a string and creates a new copy of
  it but with all instances of a specified character removed from it.\\
  \mintinline{c}{char * removeCharCopy(const char *s, char c);}\\
  Take care that the new copy does not waste memory.

  \item Write a function that takes a string and splits it up to an 
  \emph{array} of strings.  The split will be length-based: the function 
  will also take an integer $n$ and will split the given string up into 
  strings of length $n$.  It is possible that the last string will not 
  be of length $n$.  You will not need to communicate how large the 
  resulting array is as the calling function knows the string length 
  and $n$.\\
  \mintinline{c}{char **lengthSplit(const char *s, int n);}
  For example, if we pass 
  
  \mintinline{c}{"Hello World, how are you?"} with $n = 3$ then it 
  should return an array of size 9 containing the strings \mintinline{c}{"Hel"}, \mintinline{c}{"lo "}, \mintinline{c}{"Wor"}, 
\mintinline{c}{"ld,"}, \mintinline{c}{" ho"}, \mintinline{c}{"w a"}, \mintinline{c}{"re "}, \mintinline{c}{"you"}, \mintinline{c}{"?"}

\end{enumerate}

\section*{Instructions}

\begin{itemize}

  \item Place all your function prototypes into a file 
  named \mintinline{text}{string_utils.h} and and their definitions in a
  file named \mintinline{text}{string_utils.c}.  
  
  \item In addition, create a main test driver program called
  \mintinline{text}{stringTester.c} that demonstrates at least 3 cases 
  per function to verify their output.  Hand in your tester.
  
  \item You are encouraged to collaborate any number of students 
  before, during, and after your scheduled hack session.  

  \item You may (in fact are encouraged) to define any additional
  ``helper'' functions that may help you.
  
  \item Include the name(s) of everyone who worked together on
  this activity in your source file's header.

  \item Turn in all of your files via webhandin, making sure that 
  it runs and executes correctly in the webgrader.  Each individual 
  student will need to hand in their own copy and will receive 
  their own individual grade.

\end{itemize}  


\end{document}
