\documentclass[12pt]{scrartcl}


\usepackage{epsfig,amssymb}

\usepackage{xcolor}
\usepackage{graphicx}
\usepackage{epstopdf}
\usepackage{multirow}

\definecolor{darkred}{rgb}{0.5,0,0}
\definecolor{darkgreen}{rgb}{0,0.5,0}
\usepackage[pdfusetitle]{hyperref}
\hypersetup{
  letterpaper,
  colorlinks,
  linkcolor=red,
  citecolor=darkgreen,
  menucolor=darkred,
  urlcolor=blue,
  pdfpagemode=none,
}

\usepackage{fullpage}
\usepackage{tikz}
\pagestyle{empty} %
\usepackage{subfigure}

\definecolor{MyDarkBlue}{rgb}{0,0.08,0.45}
\definecolor{MyDarkRed}{rgb}{0.45,0.08,0}
\definecolor{MyDarkGreen}{rgb}{0.08,0.45,0.08}

\definecolor{mintedBackground}{rgb}{0.95,0.95,0.95}
\definecolor{mintedInlineBackground}{rgb}{.90,.90,1}

\usepackage[newfloat=true]{minted}

\setminted{mathescape,
           linenos,
           autogobble,
           frame=none,
           framesep=2mm,
           framerule=0.4pt,
           %label=foo,
           xleftmargin=2em,
           xrightmargin=0em,
           %startinline=true,  %PHP only, allow it to omit the PHP Tags *** with this option, variables using dollar sign in comments are treated as latex math
           numbersep=10pt, %gap between line numbers and start of line
           style=default} %syntax highlighting style, default is "default"

\setmintedinline{bgcolor={mintedBackground}}
%doesn't work with the above workaround:
\setminted{bgcolor={mintedBackground}}
\setminted[text]{bgcolor={mintedBackground},linenos=false,autogobble,xleftmargin=1em}
%\setminted[php]{bgcolor=mintedBackgroundPHP} %startinline=True}
\SetupFloatingEnvironment{listing}{name=Code Sample}
\SetupFloatingEnvironment{listing}{listname=List of Code Samples}

\setlength{\parindent}{0pt} %
\setlength{\parskip}{.25cm}
\newcommand{\comment}[1]{}

\usepackage{amsmath}
\usepackage{algorithm2e}
\SetKwInOut{Input}{input}
\SetKwInOut{Output}{output}
%NOTE: you can embed algorithms in solutions, but they cannot be floating objects; use [H] to make them non-floats

\usepackage{lastpage}

%\usepackage{titling}
\usepackage{fancyhdr}
\renewcommand*{\titlepagestyle}{fancy}
\pagestyle{fancy}
%\fancyhf{}
%\rhead{Computer Science I}
%\lhead{Guides and tutorials}
\renewcommand{\headrulewidth}{0.0pt}
\renewcommand{\footrulewidth}{0.4pt}
\lfoot{\Title\ -- Computer Science I}
\cfoot{~}
\rfoot{\thepage\ / \pageref*{LastPage}}


\makeatletter
\title{Hack 11.0}\let\Title\@title
\subtitle{Computer Science I\\
Encapsulation\\
{\small
\vskip1cm
Department of Computer Science \& Engineering \\
University of Nebraska--Lincoln}
\vskip-3cm}
%\author{Dr.\ Chris Bourke}
\date{~}
\makeatother

\begin{document}

\maketitle

\hrule

\section*{Introduction}

Hack session activities are small weekly programming assignments intended
to get you started on full programming assignments.  Collaboration is allowed
and, in fact, \emph{highly encouraged}.  You may start on the activity before
your hack session, but during the hack session you must either be actively 
working on this activity or \emph{helping others} work on the activity.
You are graded using the same rubric as assignments so documentation, style, 
design and correctness are all important.  

\section*{Problem Statement}

There are thousands of commercial, military, and local airports in the US and
around the world.  The International Civil Aviation Organization maintains a
database of current and inactive airports around the world.  The database 
uniquely identifies each airport by an alphanumeric GPS code.  Further, each
record contains the following pieces of data on each airport:
\begin{itemize}
  \item The name of the airport
  \item Its latitude in degrees in the range $[-90, 90]$ with negative values corresponding to the southern hemisphere
  \item Its longitude in degrees in the range $[-180, 180]$ with negative values corresponding to the western hemisphere
  \item The type of airport 
  \item Its elevation in (whole) feet above sea level
  \item Its municipality and its country
\end{itemize}

You will design a C structure to encapsulate these attributes to model an
airport record from the ICAO database.  You will also design several functions
to support your structure including factory functions, functions to 
create a string representation, print records, etc. You will also implement
several utility functions that use your structure to compute the air
distance(s) between airport locations using their latitude and longitude.
Recall that the air distance $d$ between two latitude/longitude points can be 
estimated using the Spherical Law of Cosines.

 $$d = \arccos{(\sin(\varphi_1) \cdot \sin(\varphi_2) + \cos(\varphi_1) \cos(\varphi_2) \cos(\Delta) )} \cdot R$$
where
\begin{itemize}
  \item $\varphi_1$ is the latitude of location $A$, $\varphi_2$ is the latitude of location $B$
  \item $\Delta$ is the difference between location $B$'s longitude and location $A$'s longitude
  \item $R$ is the (average) radius of the earth, 6,371 kilometers
\end{itemize}
This formula assumes that latitude and longitude are in radians 
$r$, $-\pi \leq r \leq \pi$.  To convert from degrees $d$ ($-180 \leq d \leq 180$) 
to radians $r$, you can use the simple formula:
  $$r = \frac{d}{180} \pi$$

More details have been provided in a header file, \mintinline{text}{airport.h}.
You will need to design your structure and implement all of the specified 
functions.


\section*{Instructions}

\begin{itemize}

  \item Place all of your function definitions in a source file named 
  \mintinline{text}{airport.c} and hand it in with your header file, 
  \mintinline{text}{airport.h}.  You may add any utility functions you
  wish but you must \emph{not} change any of the signatures of the required
  functions.
  
  \item In addition, you must create a main test driver program that 
  demonstrates at least 3 cases per function.  Name this file 
  \mintinline{text}{airportTester.c} and hand it in.

  \item You are encouraged to collaborate any number of students 
  before, during, and after your scheduled hack session.  

  \item You may (in fact are encouraged) to define any additional
  ``helper'' functions that may help you.

  \item Include the name(s) of everyone who worked together on
  this activity in your source file's header.

  \item Turn in all of your files via webhandin, making sure that 
  it runs and executes correctly in the webgrader.  Each individual 
  student will need to hand in their own copy and will receive 
  their own individual grade.
\end{itemize}  


\end{document}
