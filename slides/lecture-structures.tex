%%%%%%%%%%%%%%%%%%%%%%%%%%%%%%%%%%%%%%%%%%%%%%%%%%%%%
%
%  Template
%  Beamer Presentation by Chris Bourke
%
%%%%%%%%%%%%%%%%%%%%%%%%%%%%%%%%%%%%%%%%%%%%%%%%%%%%%%%%%%%%%%%%%%%%%%%

\documentclass[]{beamer}
%\documentclass[handout]{beamer}

\geometry{papersize={16cm,9cm}}

% For handout version:
%\usetheme[hideothersubsections,slidenumbers]{UNLTheme}
\usetheme[hideothersubsections]{UNLTheme}
\usepackage{amssymb}
\input{StandardCommands}
\usepackage[linesnumbered,ruled,vlined]{algorithm2e}
\SetKwComment{Comment}{//}{}
\DontPrintSemicolon
\SetKwSty{textsc} %
%\SetAlFnt{\scriptsize} %
\SetKwInOut{Input}{Input} %
\SetKwInOut{Output}{Output} %
%\setalcapskip{1em} % changed to
\SetAlCapSkip{1em}
\setlength{\algomargin}{2em} %
%\Setvlineskip{0em} % changed to:
\SetVlineSkip{0em}

\usepackage{tikz}
\usetikzlibrary{fadings}
\usetikzlibrary{shapes.geometric,shapes.symbols}
\usetikzlibrary{calc,shapes.multipart,chains,arrows}
\usetikzlibrary{arrows.meta,calc,shapes.multipart,chains,arrows}
%\usetikzlibrary{calc,shapes.multipart,chains,arrows}
%%\usetikzlibrary{backgrounds}
\usetikzlibrary{backgrounds}
\usetikzlibrary{decorations.pathreplacing}
\usetikzlibrary{decorations.pathmorphing}
\tikzset{onslide/.code args={<#1>#2}{%
  \only<#1>{\pgfkeysalso{#2}} % \pgfkeysalso doesn't change the path
}}
\tikzset{temporal/.code args={<#1>#2#3#4}{%
  \temporal<#1>{\pgfkeysalso{#2}}{\pgfkeysalso{#3}}{\pgfkeysalso{#4}} % \pgfkeysalso doesn't change the path
}}


\tikzset{
    fading speed/.code={
        \pgfmathtruncatemacro\tikz@startshading{50-(100-#1)*0.25}
        \pgfmathtruncatemacro\tikz@endshading{50+(100-#1)*0.25}
        \pgfdeclareverticalshading[%
            tikz@axis@top,tikz@axis@middle,tikz@axis@bottom%
        ]{axis#1}{100bp}{%
            color(0bp)=(tikz@axis@bottom);
            color(\tikz@startshading)=(tikz@axis@bottom);
            color(50bp)=(tikz@axis@middle);
            color(\tikz@endshading)=(tikz@axis@top);
            color(100bp)=(tikz@axis@top)
        }
        \tikzset{shading=axis#1}
    }
}

\usepackage{multirow}
\usepackage{multicol}

\definecolor{steelblue}{rgb}{0.2745,0.5098,0.7059}
\definecolor{limegreen}{RGB}{50,205,50}
\hypersetup{
    colorlinks = true,
    urlcolor = {steelblue},
    linkbordercolor = {white}
}

\definecolor{mintedBackground}{rgb}{0.95,0.95,0.95}
\definecolor{mintedInlineBackground}{rgb}{.90,.90,1}

%\usepackage{newfloat}
\usepackage{minted}
\setminted{mathescape,
               linenos,
               autogobble,
               frame=none,
               fontsize=\small,
               framesep=2mm,
               framerule=0.4pt,
               %label=foo,
               xleftmargin=2em,
               xrightmargin=0em,
               startinline=true,  %PHP only, allow it to omit the PHP Tags *** with this option, variables using dollar sign in comments are treated as latex math
               numbersep=10pt, %gap between line numbers and start of line
               style=default, %syntax highlighting style, default is "default"
               			    %gallery: http://help.farbox.com/pygments.html
			    	    %list available: pygmentize -L styles
               bgcolor=mintedBackground} %prevents breaking across pages
               
\setmintedinline{bgcolor={mintedInlineBackground}}
\setminted[text]{bgcolor={},linenos=false,autogobble,xleftmargin=1em}

\tikzstyle{decision} = [diamond, draw, fill=yellow!20, 
    text width=6em, text badly centered, node distance=5cm, inner sep=0pt]
\tikzstyle{block} = [rectangle, draw, fill=blue!20, 
    text width=5em, text centered, node distance=5cm, minimum height=4em]
\tikzstyle{action} = [rectangle, draw, fill=green!20, 
    text width=5em, text centered, rounded corners, node distance=5cm, minimum height=4em]
\tikzstyle{line} = [draw, -latex']

\title[~]{Computer Science I}
\subtitle{Encapsulation}
\author[~]{Dr.\ Chris Bourke\\ \email{cbourke@cse.unl.edu}} %
\date{~}

\begin{document}

\begin{frame}
  \titlepage
\end{frame}

\setbeamertemplate{section in toc}{\inserttocsectionnumber.~\inserttocsection}
\setbeamercolor{section in toc}{fg=black}
%\setbeamertemplate{subsection in toc}{~} %\inserttocsubsection\\}

\begin{frame}
  \frametitle{Outline}
%{\footnotesize
%\begin{NoHyper}
%  \tableofcontents[hideallsubsections]
%\end{NoHyper}
%}

\setbeamercolor{enumerate item}{bg=white,fg=black}
\setbeamercolor{item}{bg=white,fg=black}
\setbeamercolor{item projected}{bg=white,fg=black}
\setbeamercolor{enumerate subitem}{fg=red!80!black}
\setbeamertemplate{enumerate items}[default]
\begin{enumerate}
  \item Introduction
  %defining structures
  \item Using Structures
  %dot operator, arrow operator
  %factory function
  \item Structures with Functions \& Arrays
  %to string function
  %arrays of functions: several options

  %\item Exercises ?
\end{enumerate}

\end{frame}

\section{Introduction}

\begin{frame}
    \frametitle{}
    \framesubtitle{}
    
    \begin{center}
    {\Huge Part I: Introduction}\\
    {\Large ~}
    \end{center}

\end{frame}

\begin{frame}[fragile]
  \frametitle{Structures}
  \framesubtitle{}

\begin{itemize}[<+->]
  \item Built-in primitive types (\mintinline{c}{int}, 
  \mintinline{c}{double}, \mintinline{c}{char}) are limiting
  \item Not everything is \emph{simply} a number or character
  \item Real-world entities are made up of multiple aspects (data)
  \item Examples: Person, Team, Bank Account, etc.
  \item In code we can define \emph{objects} that \emph{encapsulate}
  multiple pieces of data
\end{itemize}

\end{frame}

\begin{frame}[fragile]
  \frametitle{Encapsulation}
  \framesubtitle{}

\begin{definition}
\emph{Encapsulation} is a mechanism by which multiple 
pieces of data can be grouped together.  
\end{definition}

\onslide<2->{More generally, encapsulation includes:}

\begin{itemize}%[<+->]
  \item<3-> Grouping of data
  \item<4-> Protection of data
  \item<5-> Grouping of methods that act on an object's data
\end{itemize}

\onslide<6->{C only provides \emph{weak} encapsulation (only grouping of data).}

\end{frame}

\begin{frame}[fragile]
  \frametitle{Structures}
  \framesubtitle{}

\begin{itemize}[<+->]
  \item C provides encapsulation through \emph{structures}
  \item You can define structures that group data called \emph{members}
  or \emph{fields}
  \item Demonstration
\end{itemize}

\end{frame}

\begin{frame}[fragile]
  \frametitle{Review}
  \framesubtitle{}

\begin{itemize}[<+->]
  \item Syntax: 
  \begin{itemize}
    \item \mintinline{c}{typedef struct}
    \item Opening/closing curly brackets
    \item Fields: type, name, semicolon
    \item Ends with name and semicolon
  \end{itemize}
  \item Structures may contain other structures; called \emph{composition}
  \item Order of declaration matters
  \item Structures are declared in header files
  \item Modern convention: \mintinline{c}{UpperCamelCasing} 
  for structure names, \mintinline{c}{lowerCamelCasing} for fields
\end{itemize}

\end{frame}

\section{Using Structures}

\begin{frame}
    \frametitle{}
    \framesubtitle{}
    
    \begin{center}
    {\Huge Part II: Using Structures}\\
    {\Large ~}
    \end{center}

\end{frame}

\begin{frame}[fragile]
  \frametitle{Using Structures}
  \framesubtitle{}

\begin{itemize}[<+->]
  \item Once declared, structures can be used like normal variables
  \item Declaration: \\
  \mintinline{c}{Student s;}
  \item To access members, you can use the \emph{dot operator}
  \item \mintinline{c}{s.nuid = 1234;}
  \item Demonstration
\end{itemize}

\end{frame}

\begin{frame}[fragile]
  \frametitle{Using Structures}
  \framesubtitle{}

\begin{minted}{c}
Student s;
s.nuid = 12345678;
s.firstName = (char *) malloc(sizeof(char) * 10);
strcpy(s.firstName, "Katherine");
s.lastName = (char *) malloc(sizeof(char) * 8);
strcpy(s.lastName, "Johnson");
s.gpa = 3.9;
s.dateOfBirth.year = 1918;
s.dateOfBirth.month = 9;
s.dateOfBirth.day = 26;
\end{minted}

\end{frame}

\begin{frame}[fragile]
  \frametitle{Factory Functions}
  \framesubtitle{}


\begin{itemize}[<+->]
  \item Creating instances of structures is a common task
  \item Best to create a function to facilitate the details
  \item Sometimes called \emph{factory} functions (or \emph{constructors} in Object-Oriented Programming languages)
  \item Dynamically construct an instance using \mintinline{c}{malloc()}
  \item When using pointers to structures, you can use the \emph{arrow operator}:\\
  \mintinline{c}{s->nuid}
  \item Demonstration
  %do it normally, then create a second one with alternative b-day
  %maybe one that takes a structure by ref and fills it?
\end{itemize}
  
\end{frame}

\section{Functions \& Arrays}

\begin{frame}
    \frametitle{}
    \framesubtitle{}
    
    \begin{center}
    {\Huge Part III: Structures with Functions \& Arrays}\\
    {\Large ~}
    \end{center}

\end{frame}

\subsection{Functions}

\begin{frame}[fragile]
  \frametitle{Passing Structures to Functions}
  \framesubtitle{}

\begin{itemize}[<+->]
  \item Already covered how to return (pointers) to dynamically allocated structures \emph{from} functions
  \item Straightforward to pass structures \emph{to} functions
  \item Generally want to always want to pass-by-reference
  \item Passing by value results in a (potentially) large memory copy
  \item Entire structure is copied to the call stack
  \item Pass by reference: only a pointer is copied
  \item Demonstration
\end{itemize}

\end{frame}

\begin{frame}[fragile]
  \frametitle{Demo}
  \framesubtitle{}

\begin{minted}{c}
void printStudent(const Student *s) {

  char *str = studentToString(s);
  printf("%s\n", str);
  //clean up after yourself:
  free(str);
  //printf("%s, %s (%08d), %.2f\n", s->lastName, s->firstName, s->nuid, s->gpa);
  return;
}

char * studentToString(const Student *s) {
  char buffer[1000];
  sprintf(buffer, "%s, %s (%08d), %.2f", s->lastName, s->firstName, s->nuid, s->gpa);
  char *result = (char *) malloc( (strlen(buffer)+1) * sizeof(char));
  strcpy(result, buffer);
  return result;
}
\end{minted}

\end{frame}

\subsection{Arrays}

\begin{frame}[fragile]
  \frametitle{Arrays of Structures}
  \framesubtitle{}

\begin{itemize}[<+->]
  \item Multiple structures can be stored in arrays
  \item Several ways to achieve this:
  \begin{itemize}
    \item Array of contiguous structures
    %include here a printRoster function to demonstrate
    \item Array of pointers to dynamic structures
    \item Array of pointers to contiguous structures
  \end{itemize}
  \item Demonstration
\end{itemize}

\end{frame}

\begin{frame}[fragile]
  \frametitle{Arrays of Structures}
  \framesubtitle{}

\mintinline{c}{Student *roster = (Student *) malloc(sizeof(Student) * n);}

\begin{center}
\vskip-1.25cm
%\documentclass[12pt]{scrbook}
%
%\usepackage{tikz}
%\usepackage{minted}
%\usetikzlibrary{decorations.pathreplacing,arrows}
%
%\usepackage{fullpage}
%\usepackage{subfigure}
%\begin{document}
%
%
%Lorem Ipsum is simply dummy text of the printing and typesetting industry. Lorem Ipsum has been the industry's standard dummy text ever since the 1500s, when an unknown printer took a galley of type and scrambled it to make a type specimen book. It has survived not only five centuries, but also the leap into electronic typesetting, remaining essentially unchanged. It was popularised in the 1960s with the release of Letraset sheets containing Lorem Ipsum passages, and more recently with desktop publishing software like Aldus PageMaker including versions of Lorem Ipsum.
%
%xTODO: set vertical alignment of cells
%xTODO: curve the lines
%xTODO: lines should go to the top of the box

{\setmintedinline{bgcolor={}}


\begin{tikzpicture}[scale=0.5,transform shape]

% Define block styles
\tikzstyle{box} = [rectangle,
                   draw,
                   fill=white,
                   text width=3.0cm,
                   text height=0.9cm,
                   text centered,
                   inner sep=5pt,
                   node distance=1.85cm]

\tikzstyle{line} = [draw, -latex']; %

%\draw[white] (-1, 0) rectangle (19, -1);
    \node [] (init) at (0,0) {\mintinline{c}{Student *roster}};

    \node [box, right of=init,node distance=5cm] (p0) {\mintinline{c}{roster[0]} \\
    (\mintinline{c}{Student})};
    %\node[node distance = 3cm,right of=p0] (d0) {?};
    %\draw[line] (p0) -- (d0);
    \node [box, below of=p0] (p1) {\mintinline{c}{roster[1]} \\
    (\mintinline{c}{Student})};
    %\node[node distance = 3cm,right of=p1] (d1) {?};
    %\draw[line] (p1) -- (d1);
    \node [box, below of=p1] (p2) {\mintinline{c}{roster[2]} \\
    (\mintinline{c}{Student})};
    %\node[node distance = 3cm,right of=p2] (d2) {?};
    %\draw[line] (p2) -- (d2);
    \node [below of=p2,node distance=2.0cm] (pd) {$\vdots$};
    \node [box, below of=pd,node distance=2.0cm] (pn) {\mintinline{c}{roster[n-1]} \\
    (\mintinline{c}{Student})};
    %\node[node distance = 3cm,right of=pn] (dn) {?};
    %\draw[line] (pn) -- (dn);
    \path [line] (init) -- (p0);
    
    \draw [decorate,decoration={brace,amplitude=6pt,mirror},xshift=4pt,yshift=0pt] (6.75, -0.8) -- (6.75, 0+0.9) node [align=left,black,midway,xshift=1.25cm]  {40 bytes};
    \draw [decorate,decoration={brace,amplitude=6pt,mirror},xshift=4pt,yshift=0pt] (6.75, -0.8-1.9) -- (6.75, 0+0.9-1.9) node [align=left,black,midway,xshift=1.25cm]  {40 bytes};
    \draw [decorate,decoration={brace,amplitude=6pt,mirror},xshift=4pt,yshift=0pt] (6.75, -0.8-1.9-1.9) -- (6.75, 0+0.9-1.9-1.9) node [align=left,black,midway,xshift=1.25cm]  {40 bytes};

    \draw [decorate,decoration={brace,amplitude=6pt,mirror},xshift=4pt,yshift=0pt] (6.75, -0.8-1.9-1.9-4) -- (6.75, 0+0.9-1.9-1.9-4) node [align=left,black,midway,xshift=1.25cm]  {40 bytes};


\end{tikzpicture}

%\caption[Contiguous Structure Array]{An array of structures.  Each record is stored in a contiguous manner one after the other.}
%\label{figure:arrayOfStructures}
%\end{figure}
}

%\end{document}


\end{center}  
  
\end{frame}

\begin{frame}[fragile]
  \frametitle{Arrays of Structure Pointers}
  \framesubtitle{}

\begin{minted}{c}
Student **roster = (Student **) malloc(sizeof(Student*) * n);
...
roster[i] = (Student *) malloc(sizeof(Student));
\end{minted}
\begin{center}
\vskip-1.5cm
{\setmintedinline{bgcolor={}}


\begin{tikzpicture}[scale=0.4,transform shape]

% Define block styles
\tikzstyle{box} = [rectangle,
                   draw,
                   fill=white,
                   text width=3.0cm,
                   %text height=0.9cm,
                   text centered,
                   inner sep=5pt,
                   node distance=1.25cm]

\tikzstyle{struct} = [rectangle,
                   draw,
                   fill=white,
                   text width=2.5cm,
                   minimum height=2.0cm,
                   text centered,
                   inner sep=5pt,
                   node distance=1.25cm]

\tikzstyle{line} = [draw, -latex']; %

%\draw[green] (-1, 0) rectangle (19, -1);
    \node [] (init) at (0,0) {\mintinline{c}{Student **roster}};

    \node [box, right of=init,node distance=5cm] (p0) {\mintinline{c}{roster[0]} \\
    (\mintinline{c}{Student*})};
    %\node[node distance = 3cm,right of=p0] (d0) {?};
    %\draw[line] (p0) -- (d0);
    \node [box, below of=p0] (p1) {\mintinline{c}{roster[1]} \\
    (\mintinline{c}{Student*})};
    %\node[node distance = 3cm,right of=p1] (d1) {?};
    %\draw[line] (p1) -- (d1);
    \node [box, below of=p1] (p2) {\mintinline{c}{roster[2]} \\
    (\mintinline{c}{Student*})};
    %\node[node distance = 3cm,right of=p2] (d2) {?};
    %\draw[line] (p2) -- (d2);
    \node [below of=p2,node distance=1.5cm] (pd) {$\vdots$};
    \node [box, below of=pd,node distance=1.5cm] (pn) {\mintinline{c}{roster[n-1]} \\
    (\mintinline{c}{Student*})};
    %\node[node distance = 3cm,right of=pn] (dn) {?};
    %\draw[line] (pn) -- (dn);
    \path [line] (init) -- (p0);
    

    \node[struct] (a) at (11, 2) {\mintinline{c}{Student} \\(40 bytes)};
    %\draw[line] (p0.east) -- (a.west);
    
    \draw[line] (node cs:name=p0,angle=0)
      .. controls +(east:1cm) and +(-2,0) .. (a.155);

    \node[struct] (b) at (12, -1) {\mintinline{c}{Student} \\(40 bytes)};
    \draw[line] (node cs:name=p1,angle=0)
      .. controls +(east:3cm) and +(-1,0) .. (b.155);

    \node[struct] (c) at (10, -4) {\mintinline{c}{Student} \\(40 bytes)};
    \draw[line] (node cs:name=p2,angle=0)
      .. controls +(east:1cm) and +(-2,0) .. (c.155);


    \node[struct] (d) at (13, -7) {\mintinline{c}{Student} \\(40 bytes)};
    \draw[line] (node cs:name=pn,angle=0)
      .. controls +(east:4cm) and +(-2,0) .. (d.155);

\end{tikzpicture}
%
%\caption[Array of Structure Pointers]{An array of structure pointers.  Each record is a pointer that refers to a structure which may be stored non-contiguously in completely different memory locations.}
%\label{figure:arrayOfStructurePointers}
%\end{figure}
}

\end{center}  
  
\end{frame}

\begin{frame}[fragile]
  \frametitle{Hybrid Solution}
  \framesubtitle{}

\begin{minted}[fontsize=\scriptsize]{c}
Student **roster = (Student **) malloc(sizeof(Student*) * n);
Student *rosterData = (Student *) malloc(sizeof(Student) * n);
...
roster[i] = &rosterData[i];
\end{minted}

\vskip-1.0cm
%\documentclass[12pt]{scrbook}
%
%\usepackage{tikz}
%\usepackage{minted}
%\usetikzlibrary{decorations.pathreplacing,arrows}
%
%\usepackage{fullpage}
%\usepackage{subfigure}
%\begin{document}
%
%
%Lorem Ipsum is simply dummy text of the printing and typesetting industry. Lorem Ipsum has been the industry's standard dummy text ever since the 1500s, when an unknown printer took a galley of type and scrambled it to make a type specimen book. It has survived not only five centuries, but also the leap into electronic typesetting, remaining essentially unchanged. It was popularised in the 1960s with the release of Letraset sheets containing Lorem Ipsum passages, and more recently with desktop publishing software like Aldus PageMaker including versions of Lorem Ipsum.
%
%xTODO: set vertical alignment of cells
%xTODO: curve the lines
%xTODO: lines should go to the top of the box

{\setmintedinline{bgcolor={}}


\begin{tikzpicture}[scale=0.45,transform shape]

% Define block styles
\tikzstyle{box} = [rectangle,
                   draw,
                   fill=white,
                   text width=3.0cm,
                   text height=0.9cm,
                   text centered,
                   inner sep=5pt,
                   node distance=1.85cm]

\tikzstyle{pbox} = [rectangle,
                   draw,
                   fill=white,
                   text width=3.0cm,
                   text height=0.3cm,
                   text centered,
                   inner sep=5pt,
                   node distance=1.25cm]

\tikzstyle{line} = [draw, -latex']; %

    \node [] (init) at (0,-1) {\mintinline{c}{Student **roster}};

    \node [pbox,right of=init,node distance=5cm] (p0) {\mintinline{c}{roster[0]} \\
    (\mintinline{c}{Student*})};
    %\node[node distance = 3cm,right of=p0] (d0) {?};
    %\draw[line] (p0) -- (d0);
    \node [pbox, below of=p0] (p1) {\mintinline{c}{roster[1]} \\
    (\mintinline{c}{Student*})};
    %\node[node distance = 3cm,right of=p1] (d1) {?};
    %\draw[line] (p1) -- (d1);
    \node [pbox, below of=p1] (p2) {\mintinline{c}{roster[2]} \\
    (\mintinline{c}{Student*})};
    %\node[node distance = 3cm,right of=p2] (d2) {?};
    %\draw[line] (p2) -- (d2);
    \node [below of=p2,node distance=1.25cm] (pd) {$\vdots$};
    \node [pbox, below of=pd,node distance=1.5cm] (pn) {\mintinline{c}{roster[n-1]} \\
    (\mintinline{c}{Student*})};
    %\node[node distance = 3cm,right of=pn] (dn) {?};
    %\draw[line] (pn) -- (dn);
    \path [line] (init) -- (p0);
    

    \node [] (init) at (13,2) {\mintinline{c}{Student *rosterData}};

    \node [box, below of=init,node distance=2cm] (d0) {\mintinline{c}{roster[0]} \\
    (\mintinline{c}{Student})};
    %\node[node distance = 3cm,right of=p0] (d0) {?};
    %\draw[line] (p0) -- (d0);
    \node [box, below of=d0] (d1) {\mintinline{c}{roster[1]} \\
    (\mintinline{c}{Student})};
    %\node[node distance = 3cm,right of=p1] (d1) {?};
    %\draw[line] (p1) -- (d1);
    \node [box, below of=d1] (d2) {\mintinline{c}{roster[2]} \\
    (\mintinline{c}{Student})};
    %\node[node distance = 3cm,right of=p2] (d2) {?};
    %\draw[line] (p2) -- (d2);
    \node [below of=d2,node distance=2.0cm] (pdd) {$\vdots$};
    \node [box, below of=pdd,node distance=2.0cm] (dn) {\mintinline{c}{roster[n-1]} \\
    (\mintinline{c}{Student})};
    %\node[node distance = 3cm,right of=pn] (dn) {?};
    %\draw[line] (pn) -- (dn);
    \path [line] (init) -- (d0);
    
    \draw [decorate,decoration={brace,amplitude=6pt,mirror},xshift=4pt,yshift=0pt] (6.75+8, -0.8) -- (6.75+8, 0+0.9) node [align=left,black,midway,xshift=1.25cm]  {40 bytes};
    \draw [decorate,decoration={brace,amplitude=6pt,mirror},xshift=4pt,yshift=0pt] (6.75+8, -0.8-1.9) -- (6.75+8, 0+0.9-1.9) node [align=left,black,midway,xshift=1.25cm]  {40 bytes};
    \draw [decorate,decoration={brace,amplitude=6pt,mirror},xshift=4pt,yshift=0pt] (6.75+8, -0.8-1.9-1.9) -- (6.75+8, 0+0.9-1.9-1.9) node [align=left,black,midway,xshift=1.25cm]  {40 bytes};

    \draw [decorate,decoration={brace,amplitude=6pt,mirror},xshift=4pt,yshift=0pt] (6.75+8, -0.8-1.9-1.9-4) -- (6.75+8, 0+0.9-1.9-1.9-4) node [align=left,black,midway,xshift=1.25cm]  {40 bytes};

    \draw [decorate,decoration={brace,amplitude=6pt},xshift=4pt,yshift=0pt] 
      (6.75-3.75, -0.8-1.9-1.9-2.5+.25) -- (6.75-3.75, 0+0.9-1.9-1.9-2.5-.25) node [align=left,black,midway,xshift=-1.1cm]  {8 bytes};

%\draw[->] (p0.east) -- (d0.west);
%\draw[->] (p1.east) -- (d1.west);
%\draw[->] (p2.east) -- (d2.west);
%\draw[->] (pn.east) -- (dn.west);

\draw[line] (node cs:name=p0,angle=0)
      .. controls +(east:1cm) and +(-2,0) .. (d0.160);

\draw[line] (node cs:name=p1,angle=0)
      .. controls +(east:1cm) and +(-2,0) .. (d1.160);

\draw[line] (node cs:name=p2,angle=0)
      .. controls +(east:1cm) and +(-2,0) .. (d2.160);

\draw[line] (node cs:name=pn,angle=0)
      .. controls +(east:1cm) and +(-2,0) .. (dn.160);

\end{tikzpicture}

%\caption[Hybrid Array of Structures]{Hybrid Array of Structures.  The \mintinline{c}{rosterData} is an array of contiguous structures.  The \mintinline{c}{roster} 
%array is an array of pointers that refer to each record.  Accessing elements in
%\mintinline{c}{rosterData} is done indirectly through a pointer in \mintinline{c}{roster}}
%\label{figure:arrayOfStructuresHybrid}
%\end{figure}
}




%\end{document}


  
\end{frame}


\end{document} 
